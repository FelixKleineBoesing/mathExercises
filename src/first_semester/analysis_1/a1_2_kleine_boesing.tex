%! Author = felix
%! Date = 14.10.2024

% Preamble
\documentclass[11pt]{article}

% Packages
\usepackage{amsmath}
\usepackage{amsfonts}

% Document
\begin{document}

\title{Übungsblatt 2}
\author{Felix Kleine Bösing}
\maketitle

\section*{Aufgabe 1}

Zeigen Sie: Für alle \( n \in \mathbb{N} \) gilt
\[
\sum_{k=1}^{n} k^3 = \left( \sum_{k=1}^{n} k \right)^2
\]

\subsection*{Beweis durch vollständige Induktion}

\textbf{Induktionsanfang:} Für \( n = 1 \) gilt:
\[
\sum_{k=1}^{1} k^3 = 1^3 = 1
\]
und
\[
\left( \sum_{k=1}^{1} k \right)^2 = (1)^2 = 1.
\]
Also gilt die Aussage für \( n = 1 \).

\textbf{Induktionsvoraussetzung:}
Angenommen, die Aussage gilt für ein \( n \in \mathbb{N} \), d.h.
\[
\sum_{k=1}^{n} k^3 = \left( \sum_{k=1}^{n} k \right)^2.
\]

\textbf{Induktionsschritt:}
Es ist zu zeigen, dass die Aussage auch für \( n + 1 \) gilt:
\[
\sum_{k=1}^{n+1} k^3 = \left( \sum_{k=1}^{n+1} k \right)^2.
\]
Die linke Seite der Gleichung ist:
\[
\sum_{k=1}^{n+1} k^3 = \sum_{k=1}^{n} k^3 + (n+1)^3.
\]
Nach Induktionsvoraussetzung gilt:
\[
\sum_{k=1}^{n} k^3 = \left( \sum_{k=1}^{n} k \right)^2.
\]
Daher folgt:
\[
\sum_{k=1}^{n+1} k^3 = \left( \sum_{k=1}^{n} k \right)^2 + (n+1)^3.
\]

Nun betrachten wir die rechte Seite der Gleichung:
\[
\left( \sum_{k=1}^{n+1} k \right)^2 = \left( \sum_{k=1}^{n} k + (n+1) \right)^2 = \left( S_n + (n+1) \right)^2,
\]
wobei \( S_n = \sum_{k=1}^{n} k \) ist.

Erweitern wir den Term:
\[
\left( S_n + (n+1) \right)^2 = S_n^2 + 2S_n(n+1) + (n+1)^2.
\]

Vergleichen wir dies mit der linken Seite:
\[
S_n^2 + (n+1)^3 = S_n^2 + 2S_n(n+1) + (n+1)^2.
\]

Beide Seiten stimmen überein, also gilt die Aussage auch für \( n + 1 \).

\textbf{Schluss:}
Da die Aussage für \( n = 1 \) gilt und der Induktionsschritt erfolgreich durchgeführt wurde, folgt aus der vollständigen Induktion, dass die Gleichung für alle \( n \in \mathbb{N} \) gilt:
\[
\sum_{k=1}^{n} k^3 = \left( \sum_{k=1}^{n} k \right)^2.
\]

\section*{Aufgabe 2}

Geben Sie je ein Beispiel für eine Abbildung von \( \mathbb{R} \) nach \( \mathbb{R} \), welche

\begin{itemize}
    \item[a)] injektiv und surjektiv ist,
    \item[b)] injektiv, aber nicht surjektiv ist,
    \item[c)] surjektiv, aber nicht injektiv ist,
    \item[d)] weder injektiv noch surjektiv ist.
\end{itemize}

\subsection*{Lösung}

\subsubsection*{(a) Injektiv und sujektiv (bijektiv)}

Ein Beispiel für eine bijektive Abbildung ist:
\[
f(x) = x
\]
\textbf{Begründung:}
\begin{itemize}
    \item Injektivität: Wenn \( f(x_1) = f(x_2) \), dann folgt sofort \( x_1 = x_2 \). Also ist \( f \) injektiv.
    \item Surjektivität: Für jedes \( y \in \mathbb{R} \) gibt es ein \( x \in \mathbb{R} \) mit \( f(x) = y \), nämlich \( x = y \). Also ist \( f \) auch surjektiv.
\end{itemize}

\subsubsection*{(b) Injektiv, aber nicht surjektiv}

Ein Beispiel für eine injektive, aber nicht surjektive Abbildung ist:
\[
f(x) = e^x
\]
\textbf{Begründung:}
\begin{itemize}
    \item Injektivität: Wenn \( e^{x_1} = e^{x_2} \), folgt \( x_1 = x_2 \), also ist \( f \) injektiv.
    \item Nicht surjektiv: Es gibt kein \( x \in \mathbb{R} \), für das \( f(x) = -1 \) gilt, da der Wertebereich von \( e^x \) nur positive Werte annimmt. Daher ist \( f \) nicht surjektiv.
\end{itemize}

\subsubsection*{(c) Surjektiv, aber nicht injektiv}

Ein Beispiel für eine surjektive, aber nicht injektive Abbildung ist:
\[
f(x) = x^3
\]
\textbf{Begründung:}
\begin{itemize}
    \item Surjektivität: Für jedes \( y \in \mathbb{R} \) gibt es ein \( x \in \mathbb{R} \), für das \( f(x) = y \), nämlich \( x = \sqrt[3]{y} \). Also ist \( f \) surjektiv.
    \item Nicht injektiv: Es gibt verschiedene Werte \( x \), die denselben Funktionswert haben, zum Beispiel \( f(-1) = (-1)^3 = -1 \) und \( f(1) = 1^3 = 1 \). Also ist \( f \) nicht injektiv.
\end{itemize}

\subsubsection*{(d) Weder injektiv noch surjektiv}

Ein Beispiel fpr eine Abbildung, die weder injektiv noch surjektiv ist:
\[
f(x) = \sin(x)
\]
\textbf{Begründung:}
\begin{itemize}
    \item Nicht injektiv: \( \sin(x) \) ist nicht injektiv, da \( \sin(0) = \sin(2\pi) = 0 \). Es gibt also mehrere Werte \( x \), die denselben Funktionswert haben.
    \item Nicht surjektiv: \( \sin(x) \) nimmt nur Werte im Intervall \( [-1, 1] \) an, also gibt es kein \( x \in \mathbb{R} \), für das \( f(x) = 2 \) gilt. Daher ist \( f \) nicht surjektiv.
\end{itemize}

\section*{Aufgabe 3}

Sei \( (K, +, \cdot, \leq) \) ein angeordneter Körper und \( A \subseteq K \) eine nach oben beschränkte Teilmenge.

\subsection*{(a) Besitzt \( A \) ein Supremum \( s \), so ist \( s \) eindeutig bestimmt.}

\textbf{Beweis:} Das Supremum einer Menge \( A \) ist das kleinste Element, das eine obere Schranke für \( A \) ist. Angenommen, es gäbe zwei Suprema \( s_1 \) und \( s_2 \). Da beide Suprema obere Schranken sind, gilt für alle \( a \in A \):
\[
a \leq s_1 \quad \text{und} \quad a \leq s_2.
\]
Da \( s_1 \) und \( s_2 \) Suprema sind, gilt \( s_1 \leq s_2 \) und \( s_2 \leq s_1 \), also \( s_1 = s_2 \). Daher ist das Supremum eindeutig bestimmt.

\subsection*{(b) Besitzt \( A \) ein Maximum \( m \in A \), so ist \( m \) eindeutig bestimmt.}

\textbf{Beweis:} Ein Maximum \( m \in A \) ist das größte Element in \( A \). Angenommen, \( m_1 \) und \( m_2 \) seien zwei Maxima. Dann gilt:
\[
a \leq m_1 \quad \text{und} \quad a \leq m_2.
\]
Da beide Maxima sind, folgt \( m_1 \leq m_2 \) und \( m_2 \leq m_1 \), also \( m_1 = m_2 \). Daher ist das Maximum eindeutig bestimmt.

\section*{Aufgabe 4}

Seien \( A := [0,1] \), \( B := (-\infty, 0) \) und \( M := (-\infty, 0) \cup (0, \infty) \) Teilmengen von \( \mathbb{R} \).

\subsection*{(a) Es gilt \( \sup(A) = 1 \in \mathbb{R} \).}

\textbf{Begründung:} Die Menge \( A = [0,1] \) ist eine beschränkte Teilmenge von \( \mathbb{R} \). Da 1 das größte Element in \( A \) ist, ist \( 1 \) das Supremum von \( A \):
\[
\sup(A) = 1.
\]

\subsection*{(b) Es gilt \( \sup(B) = 0 \in \mathbb{R} \).}

\textbf{Begründung:} Die Menge \( B = (-\infty, 0) \) ist nach oben beschränkt, aber 0 selbst ist nicht in \( B \) enthalten. Da 0 die kleinste obere Schranke von \( B \) ist:
\[
\sup(B) = 0.
\]

\subsection*{(c) Die Ordnung \( \leq \) auf \( \mathbb{R} \) induziert eine Ordnung \( \leq_M \) auf \( M \).}

\textbf{Begründung:} Da \( M = (-\infty, 0) \cup (0, \infty) \subset \mathbb{R} \), wird die Ordnung \( \leq \) auf \( \mathbb{R} \) auch auf \( M \) übertragen. Für alle \( x, y \in M \) gilt:
\[
x \leq y \implies x \leq_M y.
\]
Die Ordnung \( \leq_M \) ist also durch die Ordnung auf \( \mathbb{R} \) induziert.

\subsection*{(d) \( B \) besitzt kein Supremum in \( M \).}

\textbf{Begründung:} Da \( M = (-\infty, 0) \cup (0, \infty) \) an der Stelle \( 0 \) „getrennt“ ist, gehört 0 nicht zu \( M \). Daher kann 0 nicht das Supremum von \( B \) in \( M \) sein, und keine andere Zahl in \( M \) erfüllt diese Rolle. Also besitzt \( B \) kein Supremum in \( M \).

\end{document}
