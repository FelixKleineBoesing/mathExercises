%! Author = felix
%! Date = 14.10.2024

% Preamble
\documentclass[11pt]{article}

% Packages
\usepackage{amsmath}
\usepackage{amsfonts}

% Document
\begin{document}

\title{Übungsblatt 3}
\author{Felix Kleine Bösing}
\maketitle

\section*{Aufgabe 1}

\subsection*{Teil (a)}
Zeigen Sie, dass \( \left( \sqrt[n]{x} \right)^n = x \).

\textbf{Beweis:} Nach der Definition der \( n \)-ten Wurzel ist \( \sqrt[n]{x} = y \) genau dann, wenn \( y^n = x \). Daher folgt:
\[
\left( \sqrt[n]{x} \right)^n = x.
\]

\subsection*{Teil (b)}
Zeigen Sie, dass \( \{ y \in \mathbb{R}_+ : y^n = x \} = \{ \sqrt[n]{x} \} \).

\textbf{Beweis:} Da \( \sqrt[n]{x} \) die einzige positive reelle Zahl ist, die \( y^n = x \) erfüllt, folgt:
\[
\{ y \in \mathbb{R}_+ : y^n = x \} = \{ \sqrt[n]{x} \}.
\]

\subsection*{Teil (c)}
Zeigen Sie \( x^p \cdot x^q = x^{p+q} \) und \( (x^p)^q = x^{p \cdot q} \).

\textbf{Beweis:}

1. Für \( x^p \cdot x^q = x^{p+q} \):
   \[
   x^p = \prod_{i=1}^{p} x, \quad x^q = \prod_{j=1}^{q} x.
   \]
   Wenn wir diese Produkte multiplizieren, ergibt sich:
   \[
   x^p \cdot x^q = \left( \prod_{i=1}^{p} x \right) \cdot \left( \prod_{j=1}^{q} x \right) = \prod_{k=1}^{p+q} x = x^{p+q}.
   \]

2. Für \( (x^p)^q = x^{p \cdot q} \):
   \[
   x^p = \prod_{i=1}^{p} x.
   \]
   Dann gilt:
   \[
   (x^p)^q = \prod_{j=1}^{q} \left( \prod_{i=1}^{p} x \right) = \prod_{k=1}^{p \cdot q} x = x^{p \cdot q}.
   \]

\subsection*{Teil (d)}
Zeigen Sie \( (xy)^p = x^p y^p \) und \( \left( \frac{x}{y} \right)^p = \frac{x^p}{y^p} \).

\textbf{Beweis:}

1. Für \( (xy)^p = x^p y^p \):
   \[
   (xy)^p = \prod_{i=1}^{p} (xy) = \left( \prod_{i=1}^{p} x \right) \cdot \left( \prod_{i=1}^{p} y \right) = x^p y^p.
   \]

2. Für \( \left( \frac{x}{y} \right)^p = \frac{x^p}{y^p} \):
   \[
   \left( \frac{x}{y} \right)^p = \prod_{i=1}^{p} \frac{x}{y} = \frac{\prod_{i=1}^{p} x}{\prod_{i=1}^{p} y} = \frac{x^p}{y^p}.
   \]

\subsection*{Teil (e)}
Zeigen Sie \( x < y \land p > 0 \Rightarrow x^p < y^p \).

\textbf{Beweis:} Wenn \(x < y\) und \(p>0\), dann bleibt bei der Potenzierung die Ordnung erhalten, da die Funktion \(f(t)=t^p\) für \(p > 0\) monoton wachsend ist, was direkt auf den Definitionen von \(sup\) basiert.
\[
x^p < y^p.
\]

\subsection*{Teil (f)}
Zeigen Sie \( x < y \land p < 0 \Rightarrow x^p > y^p \).

\textbf{Beweis:} Da \( p < 0 \), kehrt sich die Ordnung beim Potenzieren um, da die Funktion \(f(t)=t^p\) für \(p < 0\) monoton fallend ist. Daher folgt äquivalent zu Teil (e):
\[
x^p > y^p.
\]

\subsection*{Teil (g)}
Zeigen Sie \( p < q \land x > 1 \Rightarrow x^p < x^q \).

\textbf{Beweis:} Da \( x > 1 \), \( p < q \) und \(f(t)=t^p\) für wächst \( x \) schneller bei \( q \), also:
\[
x^p < x^q.
\]

\subsection*{Teil (h)}
Zeigen Sie \( p < q \land x < 1 \Rightarrow x^p > x^q \).

\textbf{Beweis:} Da \( x < 1 \), kehrt sich die Ordnung bei höheren Exponenten um, daher:
\[
x^p > x^q.
\]

\section*{Aufgabe 2}

Seien \( a, b \in \mathbb{R} \) mit \( a, b \geq 0 \).

\subsection*{Teil (a)}
Zeigen Sie die Ungleichung \( \sqrt{ab} \leq \frac{a + b}{2} \).

\textbf{Beweis:} Wir beginnen, indem wir die Ungleichung umformen. Multiplizieren beider Seiten mit 2 ergibt:
\[
2\sqrt{ab} \leq a + b.
\]
Da \( a \) und \( b \) nicht-negativ sind, können wir beide Seiten quadrieren, ohne die Ungleichung zu verändern:
\[
(2\sqrt{ab})^2 \leq (a + b)^2,
\]
was sich vereinfacht zu:
\[
4ab \leq a^2 + 2ab + b^2.
\]
Durch Subtraktion von \( 4ab \) auf beiden Seiten erhalten wir:
\[
0 \leq a^2 - 2ab + b^2.
\]
Dies können wir als Quadrat schreiben:
\[
0 \leq (a - b)^2.
\]
Da \( (a - b)^2 \geq 0 \) immer wahr ist, folgt die gewünschte Ungleichung:
\[
\sqrt{ab} \leq \frac{a + b}{2}.
\]

\subsection*{Teil (b)}
Zeigen Sie, dass in der Ungleichung \( \sqrt{ab} \leq \frac{a + b}{2} \) genau dann Gleichheit eintritt, wenn \( a = b \).

\textbf{Beweis:} Wir setzen \( a = b \) in die Ungleichung ein und prüfen, ob dann Gleichheit gilt.

Die linke Seite der Ungleichung wird zu:
\[
\sqrt{ab} = \sqrt{a \cdot a} = \sqrt{a^2} = a.
\]

Die rechte Seite der Ungleichung wird zu:
\[
\frac{a + b}{2} = \frac{a + a}{2} = \frac{2a}{2} = a.
\]

Damit ergibt sich die Gleichung:
\[
a = a,
\]
die offensichtlich wahr ist. Dies zeigt, dass Gleichheit genau dann eintritt, wenn \( a = b \).

   \section*{Aufgabe 3}

Auf der Menge \( \mathbb{R} \times \mathbb{R} \) seien folgende Verknüpfungen \( + \) und \( \cdot \) definiert:
\[
(a, b) + (a', b') := (a + a', b + b')
\]
\[
(a, b) \cdot (a', b') := (aa' - bb', ab' + a'b).
\]

\subsection*{(a) Zeigen Sie, dass \( (\mathbb{R} \times \mathbb{R}, +, \cdot) \) ein Körper ist.}

\textbf{Beweis:} Um zu zeigen, dass \( (\mathbb{R} \times \mathbb{R}, +, \cdot) \) ein Körper ist, müssen wir folgende Eigenschaften nachweisen: Abgeschlossenheit der Operationen, Assoziativität und Kommutativität der Addition und Multiplikation, Existenz neutraler und inverser Elemente sowie das Distributivgesetz.

\begin{enumerate}
   \item \textbf{Abgeschlossenheit der Addition und Multiplikation:} \\
   Sei \( (a, b), (a', b') \in \mathbb{R} \times \mathbb{R} \).

   Für die Addition gilt:
   \[
   (a, b) + (a', b') = (a + a', b + b').
   \]
   Da \( a, a' \in \mathbb{R} \) und \( b, b' \in \mathbb{R} \), ist auch \( a + a' \in \mathbb{R} \) und \( b + b' \in \mathbb{R} \). Somit liegt \( (a + a', b + b') \in \mathbb{R} \times \mathbb{R} \), und die Addition ist abgeschlossen.

   Für die Multiplikation gilt:
   \[
   (a, b) \cdot (a', b') = (aa' - bb', ab' + a'b).
   \]
   Da \( a, a', b, b' \in \mathbb{R} \), sind auch \( aa' - bb' \in \mathbb{R} \) und \( ab' + a'b \in \mathbb{R} \). Somit ist das Ergebnis der Multiplikation wieder ein Element in \( \mathbb{R} \times \mathbb{R} \), und die Multiplikation ist abgeschlossen.

   \item \textbf{Assoziativität und Kommutativität der Addition:} \\
   Die Addition erfolgt komponentenweise, und da \( \mathbb{R} \) unter Addition assoziativ und kommutativ ist, gilt:
   \begin{align}
   ((a, b) + (a', b')) + (a'', b'') = (a + a', b + b') + (a'', b'') \\
   = (a + a' + a'', b + b' + b'') = (a, b) + ((a', b') + (a'', b''))
   \end{align}]
   also ist die Addition assoziativ.

   Außerdem ist die Addition kommutativ, da
   \[
   (a, b) + (a', b') = (a + a', b + b') = (a', b') + (a, b).
   \]

   \item \textbf{Neutrales Element der Addition:} \\
   Das neutrale Element der Addition ist \( (0, 0) \), da
   \[
   (a, b) + (0, 0) = (a + 0, b + 0) = (a, b).
   \]

   \item \textbf{Additives Inverses:} \\
   Für jedes \( (a, b) \in \mathbb{R} \times \mathbb{R} \) ist das additive Inverse gegeben durch \( (-a, -b) \), da
   \[
   (a, b) + (-a, -b) = (a - a, b - b) = (0, 0).
   \]

   \item \textbf{Kommutativität und Assoziativität der Multiplikation:} \\
   Die Kommutativität der Multiplikation folgt daraus, dass
   \[
   (a, b) \cdot (a', b') = (aa' - bb', ab' + a'b) = (a', b') \cdot (a, b).
   \]

   Für die Assoziativität der Multiplikation ist zu zeigen, dass
   \[
   ((a, b) \cdot (a', b')) \cdot (a'', b'') = (a, b) \cdot ((a', b') \cdot (a'', b''))
   \]
   was durch direkte Berechnung bestätigt werden kann. Dieser Schritt ist jedoch aufwendig und kann mit der expliziten Form der Multiplikation überprüft werden.

   \item \textbf{Neutrales Element der Multiplikation:} \\
   Das neutrale Element der Multiplikation ist \( (1, 0) \), da
   \[
   (a, b) \cdot (1, 0) = (a \cdot 1 - b \cdot 0, a \cdot 0 + b \cdot 1) = (a, b).
   \]

   \item \textbf{Multiplikatives Inverses:} \\
   Für jedes \( (a, b) \in \mathbb{R} \times \mathbb{R} \) mit \( (a, b) \neq (0, 0) \) existiert ein Inverses \( (c, d) \) mit
   \[
   (a, b) \cdot (c, d) = (1, 0).
   \]
   Durch Auflösen der Gleichung ergeben sich die Werte von \( c \) und \( d \), sodass das Inverse berechnet werden kann.

   \item \textbf{Distributivgesetz:} \\
   Die Multiplikation ist über die Addition distributiv, was sich durch die Berechnung von
   \[
   (a, b) \cdot ((a', b') + (a'', b'')) = (a, b) \cdot (a' + a'', b' + b'')
   \]
   und
   \[
   (a, b) \cdot (a', b') + (a, b) \cdot (a'', b'')
   \]
   überprüfen lässt. Beide ergeben dasselbe Resultat.
\end{enumerate}
Da alle Eigenschaften eines Körpers erfüllt sind, ist \( (\mathbb{R} \times \mathbb{R}, +, \cdot) \) ein Körper.

\subsection*{(b) Zeigen Sie, dass \( i = (0, 1) \) die Eigenschaft \( i^2 = (-1, 0) \) erfüllt.}

Wir berechnen \( i^2 \) für \( i = (0, 1) \):
\[
i \cdot i = (0, 1) \cdot (0, 1) = (0 \cdot 0 - 1 \cdot 1, 0 \cdot 1 + 1 \cdot 0) = (-1, 0).
\]
Damit ist gezeigt, dass \( i^2 = (-1, 0) \).

   \section*{Aufgabe 4}
Zeigen Sie, dass für alle \( x, y \in \mathbb{R} \) Folgendes gilt:
\[
\max\{x, y\} = \frac{1}{2} (x + y + |x - y|)
\]
und
\[
\min\{x, y\} = \frac{1}{2} (x + y - |x - y|).
\]

\textbf{Beweis:} Wir setzen \( y = x + c \) mit einer Konstante \( c \in \mathbb{R} \). Dies bedeutet, dass \( y \) um den Betrag \( c \) größer oder kleiner als \( x \) ist. Dies hilft uns dabei, den Betrag \( |x - y| \) zu analysieren und die gewünschten Ausdrücke für den Maximal- und Minimalwert zu erhalten.

### 1. Berechnung von \( |x - y| \)
Da \( y = x + c \), erhalten wir:
\[
x - y = x - (x + c) = -c.
\]
Daraus folgt:
\[
|x - y| = | - c | = |c|.
\]

Nun betrachten wir zwei Fälle, je nachdem, ob \( c \geq 0 \) oder \( c \leq 0 \) ist, um zu zeigen, dass die Formel für den Maximal- und Minimalwert unabhängig von \( c \) tatsächlich korrekt ist.

### 2. Fallunterscheidung für \( \max\{x, y\} \) und \( \min\{x, y\} \)

#### Fall 1: \( c \geq 0 \) (d.h. \( y \geq x \))

In diesem Fall ist \( y = x + c \geq x \), daher gilt \( \max\{x, y\} = y \) und \( \min\{x, y\} = x \).

Berechnung des Maximums:
\[
\max\{x, y\} = y = \frac{1}{2} (x + y + |x - y|) = \frac{1}{2} (x + (x + c) + | - c |) = \frac{1}{2} (2x + c + c) = x + c = y.
\]

Berechnung des Minimums:
\[
\min\{x, y\} = x = \frac{1}{2} (x + y - |x - y|) = \frac{1}{2} (x + (x + c) - | - c |) = \frac{1}{2} (2x + c - c) = x.
\]

#### Fall 2: \( c \leq 0 \) (d.h. \( x \geq y \))

In diesem Fall ist \( x = y - c \geq y \), daher gilt \( \max\{x, y\} = x \) und \( \min\{x, y\} = y \).

Berechnung des Maximums:
\[
\max\{x, y\} = x = \frac{1}{2} (x + y + |x - y|) = \frac{1}{2} (x + (x + c) + | - c |) = \frac{1}{2} (2x + c + c) = x + c = y.
\]


\end{document}
