%! Author = felix
%! Date = 14.10.2024

% Preamble
\documentclass[11pt]{article}

% Packages
\usepackage{amsmath}
\usepackage{amsfonts}
\usepackage{graphicx}
\DeclareMathAlphabet\mathbb{U}{fplmbb}{m}{n}
% Document
\begin{document}

\title{Übungsblatt 5}
\author{Felix Kleine Bösing, Juri Ernesto Humberg, Leonhard Meyer}
\maketitle

\section*{Aufgabe 1}
Zeigen Sie, dass für alle \( k \in \mathbb{N} \) die Folge
\(\left( \sqrt[n]{n^k} \right)_{n \in \mathbb{N}}\) gegen 1 konvergiert.

\textbf{Beweisstrategie:}
\begin{enumerate}
    \item Zeige, dass die Bedingung für \(k = 1\) gilt.
    \item Zeige, dass die Bedingung für \(k = 1\) dann für alle \(k \in \mathbb{N}\) die Bedingung impliziert.
\end{enumerate}

\textbf{Beweis für \(k = 1\):}

\(\mathrm{Z\kern-.3em\raise-0.5ex\hbox{Z}}\): \((a_{n})_{n\in\mathbb{N}} = \sqrt[n]{n} \rightarrow 1\)

Wenn die Folge \(\sqrt[n]{n}\) gegen 1 konvergiert, existiert per Definition ein \(\epsilon > 0\) zu \(N(\epsilon) \in \mathbb{N}\), so dass gilt:
\[
|\sqrt[n]{n} - 1| < \epsilon, \quad \forall n \geq N(\epsilon).
\]

Zuerst definieren wir \( x_{n} = \sqrt[n]{n} - 1 \), für welches wir nun zeigen wollen, dass \( x_{n} < \epsilon \) ist.

Um \( x_{n} \) nach \(n\) umzuformen:
\begin{align*}
    x_{n} &= \sqrt[n]{n} - 1, \\
    \sqrt[n]{n} &= x_{n} + 1, \\
    n &= (x_{n} + 1)^{n}.
\end{align*}

Wir können \((x_{n} + 1)^{n}\) mithilfe des binomischen Lehrsatzes umformen und erhalten:
\[
(x_{n} + 1)^{n} = \sum_{k=0}^{n} \binom{n}{k} x_{n}^{k}.
\]

Da wir nun eine Summe haben, können wir die Summanden für \(k = 0\) und \(k = 2\) betrachten und eine neue Ungleichung definieren, da klar ist, dass die gesamte Summe mindestens größer als die Teilsumme ist:
\[
n = \sum_{k=0}^{n} \binom{n}{k} x_{n}^{k} \geq 1 + \frac{n(n-1)}{2} x_{n}^{2}.
\]

Nun formen wir diese Ungleichung nach \(x_{n}\) um:
\[
n - 1 \geq \frac{n(n-1)}{2} x_{n}^{2},
\]
\[
\frac{2}{n} \geq x_{n}^{2},
\]
\[
x_{n} \leq \sqrt{\frac{2}{n}}.
\]

Wir wollen zeigen, dass \( x_{n} \leq \sqrt{\frac{2}{n}} < \epsilon \) für alle \(\epsilon > 0\). Dies impliziert:
\[
\sqrt{\frac{2}{n}} < \epsilon \Rightarrow \frac{2}{n} < \epsilon^{2} \Rightarrow \frac{2}{\epsilon^{2}} < n.
\]

Der Satz von Archimedes besagt, dass ein solches \(N \in \mathbb{N}\) existiert, sodass die Ungleichung erfüllt ist. Somit folgt hieraus:
\[
x_{n} = \sqrt[n]{n} - 1 \leq \sqrt{\frac{2}{n}} \leq \sqrt{\frac{2}{N}} < \epsilon.
\]

Hiermit ist die Aussage für \(k = 1\) bewiesen.

\textbf{Beweis für alle \(k \in \mathbb{N}\):}

\[
\lim\limits_{n \rightarrow \infty} \sqrt[n]{n^{k}} = \left(\lim\limits_{n \rightarrow \infty} \sqrt[n]{n}\right)^k.
\]

Da wir bereits gezeigt haben, dass jeder einzelne Ausdruck \(\lim\limits_{n \rightarrow \infty} \sqrt[n]{n}\) gegen 1 konvergiert, folgt, dass auch \(\lim\limits_{n \rightarrow \infty} \sqrt[n]{n^{k}} \rightarrow 1\).

\section*{Aufgabe 2}

\textbf{Beweis:} Wir zeigen, dass für alle \( n \in \mathbb{N} \) gilt: \( a_n \geq \sqrt{c} \) und \( a_{n+1} \leq a_n \).

\begin{enumerate}
    \item \textbf{Behauptung:} Für alle \( n \in \mathbb{N} \) gilt \( a_n \geq \sqrt{c} \).

    \item \textbf{Beweis von \( a_n \geq \sqrt{c} \):} Wir beweisen diese Aussage rekursiv. Der Startwert \( a_0 \in \mathbb{R}_+ \) ist beliebig und erfüllt \( a_0 \geq \sqrt{c} \), wenn wir \( a_0 \) entsprechend wählen. Die rekursive Definition der Folge lautet
    \[
    a_{n+1} = \frac{1}{2} \left( a_n + \frac{c}{a_n} \right).
    \]
    Angenommen, \( a_n \geq \sqrt{c} \). Dann folgt, dass \( \frac{c}{a_n} \leq \sqrt{c} \), da \( c \) positiv ist und \( a_n \geq \sqrt{c} \) angenommen wurde.

    \item \textbf{Abschätzung von \( a_{n+1} \):} Mit der rekursiven Formel können wir nun \( a_{n+1} \) abschätzen:
    \[
    a_{n+1} = \frac{1}{2} \left( a_n + \frac{c}{a_n} \right).
    \]
    Da \( a_n \geq \sqrt{c} \) und \( \frac{c}{a_n} \leq \sqrt{c} \), ergibt sich:
    \[
    a_{n+1} \geq \frac{1}{2} \left( \sqrt{c} + \sqrt{c} \right).
    \]
    Da \( \sqrt{c} + \sqrt{c} = 2 \sqrt{c} \), folgt:
    \[
    a_{n+1} \geq \frac{1}{2} \cdot 2 \sqrt{c} = \sqrt{c}.
    \]
    Damit ist gezeigt, dass \( a_{n+1} \geq \sqrt{c} \), wenn \( a_n \geq \sqrt{c} \) gilt. Folglich ist \( a_n \geq \sqrt{c} \) für alle \( n \in \mathbb{N} \).

    \item \textbf{Monotonie der Folge:} Wir zeigen, dass die Folge \( (a_n)_{n \in \mathbb{N}} \) monoton fallend ist, also \( a_{n+1} \leq a_n \).

    Es gilt:
    \[
    a_{n+1} = \frac{1}{2} \left( a_n + \frac{c}{a_n} \right).
    \]
    Da \( a_n \geq \sqrt{c} \) ist, folgt aus der Konstruktion von \( a_{n+1} \) durch das arithmetisch-geometrische Mittel, dass \( a_{n+1} \leq a_n \).
\end{enumerate}

Damit haben wir gezeigt, dass \( (a_n)_{n \in \mathbb{N}} \) eine monoton fallende und nach unten durch \( \sqrt{c} \) beschränkte Folge ist.

\subsection*{Teil (b)}

\textbf{Beweis:} Da die Folge \( (a_n)_{n \in \mathbb{N}} \) monoton fallend und nach unten durch \( \sqrt{c} \) beschränkt ist, konvergiert sie nach dem Monotoniekriterium. Sei \( a := \lim_{n \to \infty} a_n \).

Im Grenzwert folgt aus der Rekursionsgleichung
\[
a = \frac{1}{2} \left( a + \frac{c}{a} \right).
\]
Durch Umstellen ergibt sich
\[
2a = a + \frac{c}{a} \Rightarrow a = \sqrt{c}.
\]
Damit folgt \( \lim_{n \to \infty} a_n = \sqrt{c} \).

\subsection*{Teil (c)}

?

\section*{Aufgabe 3}

Berechnen Sie die Häufungspunkte, den Limes superior, sowie den Limes inferior (falls existent) der folgenden reellen Folgen.

\subsection*{(a) \( a_n = \left( \frac{3}{2} + (-1)^n \right)^n \)}

\textbf{Lösung:}
\begin{enumerate}
    \item Betrachten wir \( a_n = \left( \frac{3}{2} + (-1)^n \right)^n \).
    \item Da \( (-1)^n \) abwechselnd \( 1 \) und \( -1 \) ist, erhalten wir für gerade \( n \):
        \[
        a_{2k} = \left( \frac{3}{2} + 1 \right)^{2k} = \left( \frac{5}{2} \right)^{2k} \to \infty \quad \text{für } k \to \infty.
        \]
        Für ungerade \( n \):
        \[
        a_{2k+1} = \left( \frac{3}{2} - 1 \right)^{2k+1} = \left( \frac{1}{2} \right)^{2k+1} \to 0 \quad \text{für } k \to \infty.
        \]
    \item Daher divergiert die Folge \( a_n \), aber wir können feststellen:
        \[
        \limsup_{n \to \infty} a_n = \infty \quad \text{und} \quad \liminf_{n \to \infty} a_n = 0.
        \]
    \item Es existieren keine Häufungspunkte, da die Folge keine begrenzten Werte annimmt.
\end{enumerate}

\subsection*{(b) \( b_n = \begin{cases}
2 + \frac{1}{3n} & \text{falls } n = 3k, \\
3 + \frac{n+2}{n} & \text{falls } n = 3k + 1, \\
3 & \text{falls } n = 3k + 2
\end{cases} \)}

\textbf{Lösung:}
\begin{enumerate}
    \item Untersuchen wir die drei Fälle:
        \begin{enumerate}
            \item Für \( n = 3k \):
            \[
            b_{3k} = 2 + \frac{1}{3k} \to 2 \quad \text{für } k \to \infty.
            \]
            \item Für \( n = 3k + 1 \):
            \[
            b_{3k+1} = 3 + \frac{3k+1 + 2}{3k+1} \to 4 \quad \text{für } k \to \infty.
            \]
            \item Für \( n = 3k + 2 \):
            \[
            b_{3k+2} = 3.
            \]
        \end{enumerate}
    \item Damit haben wir die Häufungspunkte \( \{2, 3, 4\} \).
    \item Der Limes superior ist \( \limsup_{n \to \infty} b_n = 4 \) und der Limes inferior ist \( \liminf_{n \to \infty} b_n = 2 \).
\end{enumerate}

\subsection*{(c) \( c_0 = \sqrt{2} \) und \( c_{n+1} = \sqrt{2 + c_n} \) für \( n \geq 0 \)}

\textbf{Lösung:}
\begin{enumerate}
    \item Die Folge \( (c_n) \) ist monoton wachsend und nach oben beschränkt. Wir zeigen, dass sie gegen einen Grenzwert konvergiert.
    \item Sei \( L = \lim_{n \to \infty} c_n \). Dann gilt:
    \[
    L = \sqrt{2 + L}.
    \]
    \item Quadrieren beiderseits ergibt:
    \[
    L^2 = 2 + L \Rightarrow L^2 - L - 2 = 0 \Rightarrow (L - 2)(L + 1) = 0.
    \]
    \item Da \( L \geq 0 \), folgt \( L = 2 \).
    \item Somit konvergiert die Folge \( (c_n) \) gegen 2, und der einzige Häufungspunkt ist 2. Daher gilt:
    \[
    \limsup_{n \to \infty} c_n = \liminf_{n \to \infty} c_n = 2.
    \]
\end{enumerate}

\subsection*{(d) \( d_n = 42 + (-n)^n \)}

\textbf{Lösung:}
\begin{enumerate}
    \item Da \( (-n)^n \) für \( n \) gerade positiv und sehr groß wird, und für \( n \) ungerade negativ und sehr groß im Betrag, divergiert \( d_n \) abwechselnd gegen \( +\infty \) und \( -\infty \).
    \item Somit hat die Folge keinen Limes superior, keinen Limes inferior und keine Häufungspunkte.
\end{enumerate}


\section*{Aufgabe 4}

Beweisen Sie, dass eine beschränkte reelle oder komplexe Folge genau dann konvergiert, wenn sie genau einen Häufungspunkt besitzt.

\subsection*{Beweis:}

Wir beweisen die Aussage in zwei Richtungen.

\begin{enumerate}
    \item[\textbf{1. Richtung:}] \textbf{(Wenn die Folge konvergiert, hat sie genau einen Häufungspunkt)}

    Sei \( (a_n)_{n \in \mathbb{N}} \) eine beschränkte, konvergente reelle oder komplexe Folge mit Grenzwert \( L \). Da die Folge konvergiert, bedeutet dies, dass für jedes \( \epsilon > 0 \) nur endlich viele Folgenglieder außerhalb des \( \epsilon \)-Umkreises um \( L \) liegen, also
    \[
    \forall \epsilon > 0 \; \exists N \in \mathbb{N} \; \text{sodass} \; |a_n - L| < \epsilon \; \forall n \geq N.
    \]

    Da \( L \) der einzige Punkt ist, dem sich die Folge beliebig nahe annähert, ist \( L \) ein Häufungspunkt von \( (a_n) \).

    Angenommen, die Folge hätte noch einen weiteren Häufungspunkt \( L' \neq L \). Dann müsste es für \( L' \) ebenfalls ein \( \epsilon' > 0 \) geben, sodass unendlich viele Folgenglieder in dem \( \epsilon' \)-Umkreis um \( L' \) liegen. Dies widerspricht jedoch der Definition der Konvergenz, da die Folge \( (a_n) \) nur um \( L \) „häuft“. Daher kann \( L \) der einzige Häufungspunkt der Folge sein.

    Also hat eine konvergente Folge genau einen Häufungspunkt.

    \item[\textbf{2. Richtung:}] \textbf{(Wenn die Folge genau einen Häufungspunkt hat, dann konvergiert sie)}

    Sei \( (a_n)_{n \in \mathbb{N}} \) eine beschränkte Folge, die genau einen Häufungspunkt \( L \) besitzt. Da die Folge beschränkt ist, existiert nach dem Satz von Bolzano-Weierstraß eine konvergente Teilfolge \( (a_{n_k})_{k \in \mathbb{N}} \), die gegen \( L \) konvergiert, da \( L \) der einzige Häufungspunkt ist.

    Angenommen, die gesamte Folge \( (a_n) \) konvergiert nicht gegen \( L \). Dann müsste es ein \( \epsilon > 0 \) geben, sodass unendlich viele Folgenglieder \( a_n \) den \( \epsilon \)-Umkreis um \( L \) verlassen. Diese Folgenglieder könnten eine weitere Teilfolge bilden, die nicht gegen \( L \) konvergiert, was im Widerspruch dazu steht, dass \( L \) der einzige Häufungspunkt ist.

    Daher muss die gesamte Folge gegen \( L \) konvergieren.
\end{enumerate}

Damit ist gezeigt, dass eine beschränkte Folge genau dann konvergiert, wenn sie genau einen Häufungspunkt besitzt. \(\square\)


\end{document}
