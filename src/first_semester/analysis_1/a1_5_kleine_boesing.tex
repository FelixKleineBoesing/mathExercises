%! Author = felix
%! Date = 14.10.2024

% Preamble
\documentclass[11pt]{article}

% Packages
\usepackage{amsmath}
\usepackage{amsfonts}
\usepackage{graphicx}

% Document
\begin{document}

\title{Übungsblatt 5}
\author{Felix Kleine Bösing, Juri Ernesto Humberg, Leonhard Meyer}
\maketitle

\section*{Aufgabe 1}

Zeigen Sie, dass für alle \( k \in \mathbb{N} \) die Folge \( \left( \sqrt[n]{n^k} \right)_{n \in \mathbb{N}} \) gegen 1 konvergiert.

(Hinweis: Betrachten Sie zunächst den Fall \( k = 1 \).)

\subsection*{Teil (a)}

\textbf{Beweis:} Wir zeigen zunächst, dass die Folge \( \left( \sqrt[n]{n^k} \right)_{n \in \mathbb{N}} \) für \( k = 1 \) gegen 1 konvergiert.

\begin{enumerate}
    \item Betrachte die allgemeine Form der \( n \)-ten Wurzel von \( n \):
    \[
    a_n = \sqrt[n]{n} = n^{\frac{1}{n}}.
    \]

    \item Um zu zeigen, dass \( a_n \to 1 \) für \( n \to \infty \), betrachten wir den natürlichen Logarithmus von \( a_n \):
    \[
    \ln(a_n) = \ln\left(n^{\frac{1}{n}}\right) = \frac{1}{n} \ln(n).
    \]

    \item Nun untersuchen wir das Verhalten von \( \frac{\ln(n)}{n} \) für \( n \to \infty \):
    \[
    \lim_{n \to \infty} \frac{\ln(n)}{n} = 0.
    \]
    Dies folgt aus der Anwendung der Regel von de l'Hôpital, da die Ableitung des Zählers \( \ln(n) \) und die Ableitung des Nenners \( n \) wie folgt sind:
    \[
    \lim_{n \to \infty} \frac{\ln(n)}{n} = \lim_{n \to \infty} \frac{\frac{1}{n}}{1} = \lim_{n \to \infty} \frac{1}{n} = 0.
    \]

    \item Daraus folgt:
    \[
    \lim_{n \to \infty} \ln(a_n) = 0.
    \]
    Da der Logarithmus eine stetige Funktion ist, erhalten wir \( \lim_{n \to \infty} a_n = e^0 = 1 \).

\end{enumerate}

Damit ist gezeigt, dass \( \left( \sqrt[n]{n} \right)_{n \in \mathbb{N}} \) für \( k = 1 \) gegen 1 konvergiert.

\subsection*{Teil (b)}

\textbf{Beweis für allgemeines \( k \):} Nun zeigen wir die Konvergenz der Folge \( \left( \sqrt[n]{n^k} \right)_{n \in \mathbb{N}} \) für beliebiges \( k \in \mathbb{N} \).

\begin{enumerate}
    \item Die Folge hat die Form:
    \[
    a_n = \sqrt[n]{n^k} = (n^k)^{\frac{1}{n}} = n^{\frac{k}{n}}.
    \]

    \item Betrachten wir den natürlichen Logarithmus von \( a_n \):
    \[
    \ln(a_n) = \ln\left(n^{\frac{k}{n}}\right) = \frac{k}{n} \ln(n).
    \]

    \item Analog zum Fall \( k = 1 \) betrachten wir das Verhalten von \( \frac{k \ln(n)}{n} \) für \( n \to \infty \):
    \[
    \lim_{n \to \infty} \frac{k \ln(n)}{n} = k \cdot \lim_{n \to \infty} \frac{\ln(n)}{n} = k \cdot 0 = 0.
    \]

    \item Also gilt:
    \[
    \lim_{n \to \infty} \ln(a_n) = 0 \Rightarrow \lim_{n \to \infty} a_n = e^0 = 1.
    \]
\end{enumerate}

Damit haben wir gezeigt, dass die Folge \( \left( \sqrt[n]{n^k} \right)_{n \in \mathbb{N}} \) für beliebiges \( k \in \mathbb{N} \) gegen 1 konvergiert.

\section*{Aufgabe 2}

\section*{Aufgabe 3}

Berechnen Sie die Häufungspunkte, den Limes superior, sowie den Limes inferior (falls existent) der folgenden reellen Folgen.

\subsection*{(a) \( a_n = \left( \frac{3}{2} + (-1)^n \right)^n \)}

\textbf{Lösung:}
\begin{enumerate}
    \item Betrachten wir \( a_n = \left( \frac{3}{2} + (-1)^n \right)^n \).
    \item Da \( (-1)^n \) abwechselnd \( 1 \) und \( -1 \) ist, erhalten wir für gerade \( n \):
        \[
        a_{2k} = \left( \frac{3}{2} + 1 \right)^{2k} = \left( \frac{5}{2} \right)^{2k} \to \infty \quad \text{für } k \to \infty.
        \]
        Für ungerade \( n \):
        \[
        a_{2k+1} = \left( \frac{3}{2} - 1 \right)^{2k+1} = \left( \frac{1}{2} \right)^{2k+1} \to 0 \quad \text{für } k \to \infty.
        \]
    \item Daher divergiert die Folge \( a_n \), aber wir können feststellen:
        \[
        \limsup_{n \to \infty} a_n = \infty \quad \text{und} \quad \liminf_{n \to \infty} a_n = 0.
        \]
    \item Es existieren keine Häufungspunkte, da die Folge keine begrenzten Werte annimmt.
\end{enumerate}

\subsection*{(b) \( b_n = \begin{cases}
2 + \frac{1}{3n} & \text{falls } n = 3k, \\
3 + \frac{n+2}{n} & \text{falls } n = 3k + 1, \\
3 & \text{falls } n = 3k + 2
\end{cases} \)}

\textbf{Lösung:}
\begin{enumerate}
    \item Untersuchen wir die drei Fälle:
        \begin{enumerate}
            \item Für \( n = 3k \):
            \[
            b_{3k} = 2 + \frac{1}{3k} \to 2 \quad \text{für } k \to \infty.
            \]
            \item Für \( n = 3k + 1 \):
            \[
            b_{3k+1} = 3 + \frac{3k+1 + 2}{3k+1} \to 4 \quad \text{für } k \to \infty.
            \]
            \item Für \( n = 3k + 2 \):
            \[
            b_{3k+2} = 3.
            \]
        \end{enumerate}
    \item Damit haben wir die Häufungspunkte \( \{2, 3, 4\} \).
    \item Der Limes superior ist \( \limsup_{n \to \infty} b_n = 4 \) und der Limes inferior ist \( \liminf_{n \to \infty} b_n = 2 \).
\end{enumerate}

\subsection*{(c) \( c_0 = \sqrt{2} \) und \( c_{n+1} = \sqrt{2 + c_n} \) für \( n \geq 0 \)}

\textbf{Lösung:}
\begin{enumerate}
    \item Die Folge \( (c_n) \) ist monoton wachsend und nach oben beschränkt. Wir zeigen, dass sie gegen einen Grenzwert konvergiert.
    \item Sei \( L = \lim_{n \to \infty} c_n \). Dann gilt:
    \[
    L = \sqrt{2 + L}.
    \]
    \item Quadrieren beiderseits ergibt:
    \[
    L^2 = 2 + L \Rightarrow L^2 - L - 2 = 0 \Rightarrow (L - 2)(L + 1) = 0.
    \]
    \item Da \( L \geq 0 \), folgt \( L = 2 \).
    \item Somit konvergiert die Folge \( (c_n) \) gegen 2, und der einzige Häufungspunkt ist 2. Daher gilt:
    \[
    \limsup_{n \to \infty} c_n = \liminf_{n \to \infty} c_n = 2.
    \]
\end{enumerate}

\subsection*{(d) \( d_n = 42 + (-n)^n \)}

\textbf{Lösung:}
\begin{enumerate}
    \item Da \( (-n)^n \) für \( n \) gerade positiv und sehr groß wird, und für \( n \) ungerade negativ und sehr groß im Betrag, divergiert \( d_n \) abwechselnd gegen \( +\infty \) und \( -\infty \).
    \item Somit hat die Folge keinen Limes superior, keinen Limes inferior und keine Häufungspunkte.
\end{enumerate}


\section*{Aufgabe 4}

Beweisen Sie, dass eine beschränkte reelle oder komplexe Folge genau dann konvergiert, wenn sie genau einen Häufungspunkt besitzt.

\subsection*{Beweis:}

Wir beweisen die Aussage in zwei Richtungen.

\begin{enumerate}
    \item[\textbf{1. Richtung:}] \textbf{(Wenn die Folge konvergiert, hat sie genau einen Häufungspunkt)}

    Sei \( (a_n)_{n \in \mathbb{N}} \) eine beschränkte, konvergente reelle oder komplexe Folge mit Grenzwert \( L \). Da die Folge konvergiert, bedeutet dies, dass für jedes \( \epsilon > 0 \) nur endlich viele Folgenglieder außerhalb des \( \epsilon \)-Umkreises um \( L \) liegen, also
    \[
    \forall \epsilon > 0 \; \exists N \in \mathbb{N} \; \text{sodass} \; |a_n - L| < \epsilon \; \forall n \geq N.
    \]

    Da \( L \) der einzige Punkt ist, dem sich die Folge beliebig nahe annähert, ist \( L \) ein Häufungspunkt von \( (a_n) \).

    Angenommen, die Folge hätte noch einen weiteren Häufungspunkt \( L' \neq L \). Dann müsste es für \( L' \) ebenfalls ein \( \epsilon' > 0 \) geben, sodass unendlich viele Folgenglieder in dem \( \epsilon' \)-Umkreis um \( L' \) liegen. Dies widerspricht jedoch der Definition der Konvergenz, da die Folge \( (a_n) \) nur um \( L \) „häuft“. Daher kann \( L \) der einzige Häufungspunkt der Folge sein.

    Also hat eine konvergente Folge genau einen Häufungspunkt.

    \item[\textbf{2. Richtung:}] \textbf{(Wenn die Folge genau einen Häufungspunkt hat, dann konvergiert sie)}

    Sei \( (a_n)_{n \in \mathbb{N}} \) eine beschränkte Folge, die genau einen Häufungspunkt \( L \) besitzt. Da die Folge beschränkt ist, existiert nach dem Satz von Bolzano-Weierstraß eine konvergente Teilfolge \( (a_{n_k})_{k \in \mathbb{N}} \), die gegen \( L \) konvergiert, da \( L \) der einzige Häufungspunkt ist.

    Angenommen, die gesamte Folge \( (a_n) \) konvergiert nicht gegen \( L \). Dann müsste es ein \( \epsilon > 0 \) geben, sodass unendlich viele Folgenglieder \( a_n \) den \( \epsilon \)-Umkreis um \( L \) verlassen. Diese Folgenglieder könnten eine weitere Teilfolge bilden, die nicht gegen \( L \) konvergiert, was im Widerspruch dazu steht, dass \( L \) der einzige Häufungspunkt ist.

    Daher muss die gesamte Folge gegen \( L \) konvergieren.
\end{enumerate}

Damit ist gezeigt, dass eine beschränkte Folge genau dann konvergiert, wenn sie genau einen Häufungspunkt besitzt. \(\square\)


\end{document}
