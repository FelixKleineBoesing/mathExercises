%! Author = felix
%! Date = 14.10.2024

% Preamble
\documentclass[11pt]{article}

% Packages
\usepackage{amsmath}
\usepackage{amsfonts}
\usepackage{graphicx}
\usepackage{enumitem}
\DeclareMathAlphabet\mathbb{U}{fplmbb}{m}{n}
% Document
\begin{document}

\title{Übungsblatt 6}
\author{Felix Kleine Bösing, Juri Ernesto Humberg, Leonhard Meyer}
\maketitle
\newpage

\subsection*{Teil (a)}
Gegeben ist eine reelle Folge $(a_n)_{n \in \mathbb{N}}$ und eine Konstante $q \in \mathbb{R}$ mit $0 < q < 1$. Weiterhin existiert ein $N \in \mathbb{N}$, sodass
\[
|a_{n+1} - a_n| \leq q |a_n - a_{n-1}| \quad \text{für alle } n \geq N.
\]
Es ist zu zeigen, dass $(a_n)_{n \in \mathbb{N}}$ eine Cauchy-Folge ist. Außerdem soll gezeigt werden, dass diese Eigenschaft nicht notwendigerweise gilt, falls lediglich $|a_{n+1} - a_n| < |a_n - a_{n-1}|$ für alle $n \geq N$ gilt.

\textbf{Beweis:}
\begin{enumerate}
    \item \textbf{Cauchy-Folge unter der Bedingung \( |a_{n+1} - a_n| \leq q |a_n - a_{n-1}| \):}
    \begin{enumerate}
        \item Definieren wir die Abstände \( d_n = |a_{n+1} - a_n| \). Aus der Bedingung folgt, dass
        \[
        d_{n+1} \leq q \cdot d_n \quad \text{für alle } n \geq N.
        \]
        Durch wiederholte Anwendung erhalten wir für \( k \geq 1 \):
        \[
        d_{N+k} \leq q^k \cdot d_N.
        \]
        \item Für \( m > n \geq N \) gilt die Dreiecksungleichung:
        \[
        |a_m - a_n| \leq \sum_{k=n}^{m-1} |a_{k+1} - a_k| = \sum_{k=n}^{m-1} d_k.
        \]
        Setzen wir die Abschätzung \( d_k \leq q^{k-N} \cdot d_N \) ein:
        \[
        |a_m - a_n| \leq \sum_{k=n}^{m-1} q^{k-N} \cdot d_N.
        \]
        \item Die Summe der \( q^{k-N} \)-Terme bildet eine geometrische Reihe:
        \[
        \sum_{k=n}^{m-1} q^{k-N} = q^{n-N} \cdot \frac{1 - q^{m-n}}{1-q}.
        \]
        Für \( m \to \infty \) konvergiert die geometrische Reihe, und es bleibt:
        \[
        |a_m - a_n| \leq \frac{q^{n-N}}{1-q} \cdot d_N.
        \]
        \item Da \( q^{n-N} \to 0 \) für \( n \to \infty \), folgt \( |a_m - a_n| \to 0 \). Somit ist \( (a_n)_{n \in \mathbb{N}} \) eine Cauchy-Folge.
    \end{enumerate}

    \item \textbf{Die schwächere Bedingung \( |a_{n+1} - a_n| < |a_n - a_{n-1}| \):}
    \begin{enumerate}
        \item Die schwächere Bedingung \( |a_{n+1} - a_n| < |a_n - a_{n-1}| \) bedeutet nur, dass die Abstände zwischen aufeinanderfolgenden Folgengliedern kleiner werden, garantiert aber nicht, dass die Folge summierbar ist oder konvergiert.
        \item Betrachten wir die Folge \( a_n = \ln(n+1) \). Es gilt:
        \[
        |a_{n+1} - a_n| = \ln(n+2) - \ln(n+1) = \ln\left(1 + \frac{1}{n+1}\right).
        \]
        Für \( n \to \infty \) konvergiert \( \ln\left(1 + \frac{1}{n+1}\right) \to 0 \). Dennoch divergiert \( a_n \), da \( \ln(n+1) \to \infty \).
        \item Somit zeigt dieses Gegenbeispiel, dass die schwächere Bedingung \( |a_{n+1} - a_n| < |a_n - a_{n-1}| \) nicht ausreicht, um zu garantieren, dass \( (a_n)_{n \in \mathbb{N}} \) eine Cauchy-Folge ist.
    \end{enumerate}
\end{enumerate}

\subsection*{Schlussfolgerung:}
\begin{enumerate}
    \item Die Bedingung \( |a_{n+1} - a_n| \leq q |a_n - a_{n-1}| \) mit \( 0 < q < 1 \) garantiert durch die Summierbarkeit der Abstände, dass \( (a_n)_{n \in \mathbb{N}} \) eine Cauchy-Folge ist und daher konvergiert.
    \item Die schwächere Bedingung \( |a_{n+1} - a_n| < |a_n - a_{n-1}| \) ist hingegen nicht ausreichend, wie das Gegenbeispiel \( a_n = \ln(n+1) \) zeigt.
\end{enumerate}
\subsection*{Teil (b)}
Zu zeigen ist, dass die rekursiv definierten Folgen $a_n$ und $b_n$ konvergieren und deren Grenzwerte zu berechnen sind.

\textbf{Beweis:}
\begin{enumerate}
    \item \textbf{Teil (i):} Die Folge $a_n$ ist definiert durch
    \[
    a_0 = 1 \quad \text{und} \quad a_{n+1} = \frac{2 + a_n}{1 + a_n} \quad \text{für } n \geq 1.
    \]
    \begin{enumerate}
        \item \textbf{Grenzwert:} Falls die Folge konvergiert, muss der Grenzwert $a$ die Gleichung erfüllen:
        \[
        a = \frac{2 + a}{1 + a}.
        \]
        Multiplizieren mit $1+a$ ergibt:
        \[
        a (1 + a) = 2 + a \implies a^2 = 2.
        \]
        Da $a_n > 0$ für alle $n$, folgt:
        \[
        a = \sqrt{2}.
        \]

        \item \textbf{Beschränktheit der Folge:} Wir zeigen, dass die Folge $a_n$ beschränkt ist:
        \begin{itemize}
            \item Der Startwert ist $a_0 = 1$, und $a_1 = \frac{2+1}{1+1} = 1.5$.
            \item Für $a_n > 0$ gilt aus der Definition:
            \[
            a_{n+1} = \frac{2 + a_n}{1 + a_n}.
            \]
            Da der Zähler $2 + a_n$ und der Nenner $1 + a_n$ positiv sind, folgt $a_{n+1} > 0$. Weiterhin ist $a_{n+1} < 2$, da der Bruch für $a_n > 0$ kleiner als 2 bleibt.
        \end{itemize}
        Damit liegt $a_n$ im Intervall $(0, 2)$ und ist somit beschränkt.

        \item \textbf{Oszillation und Konvergenz der Folge:}
        \begin{itemize}
            \item Die Folge $a_n$ ist nicht monoton, sondern oszilliert um den Grenzwert $\sqrt{2}$. Die ersten Werte der Folge sind:
            \[
            a_0 = 1, \quad a_1 = 1.5, \quad a_2 \approx 1.4167, \quad a_3 \approx 1.413, \quad a_4 \approx 1.4142.
            \]
            \item Es ist zu sehen, dass die Folge abwechselnd über und unter $\sqrt{2}$ liegt. Insbesondere gilt:
            \[
            a_0 < \sqrt{2}, \quad a_1 > \sqrt{2}, \quad a_2 < \sqrt{2}, \quad a_3 > \sqrt{2}, \dots
            \]
        \end{itemize}

        Um die Konvergenz zu zeigen, betrachten wir die Differenzen \( |a_{n+1} - a_n| \):
        \[
        |a_{n+1} - a_n| = \left| \frac{2 + a_n}{1 + a_n} - a_n \right|.
        \]
        Bringen wir die Terme auf einen gemeinsamen Nenner:
        \[
        a_{n+1} - a_n = \frac{2 + a_n - a_n(1 + a_n)}{1 + a_n}.
        \]
        Der Zähler vereinfacht sich zu:
        \[
        2 + a_n - a_n - a_n^2 = 2 - a_n^2.
        \]
        Damit gilt:
        \[
        a_{n+1} - a_n = \frac{2 - a_n^2}{1 + a_n}.
        \]
        Für \( n \to \infty \) konvergiert \( a_n \to \sqrt{2} \), sodass \( 2 - a_n^2 \to 0 \). Der Nenner \( 1 + a_n \) bleibt positiv und beschränkt. Somit folgt:
        \[
        |a_{n+1} - a_n| \to 0.
        \]

        Da die Folge beschränkt ist und \( |a_{n+1} - a_n| \to 0 \), ist die Folge eine \textbf{Cauchy-Folge}. Da jede Cauchy-Folge im reellen Raum konvergiert, konvergiert \( (a_n) \) gegen \( \sqrt{2} \).
    \end{enumerate}
\item \textbf{Teil (ii):} Die Folge $b_n$ ist definiert durch
\[
b_0 = 1 \quad \text{und} \quad b_{n+1} = 1 + \frac{1}{b_n} \quad \text{für } n \geq 1.
\]

\begin{enumerate}
    \item \textbf{Grenzwert:} Falls die Folge $b_n$ konvergiert, existiert ein Grenzwert $b = \lim_{n \to \infty} b_n$, der die Gleichung erfüllen muss:
    \[
    b = 1 + \frac{1}{b}.
    \]
    Multiplizieren mit $b$ ergibt:
    \[
    b^2 = b + 1 \implies b^2 - b - 1 = 0.
    \]
    Die Lösung der quadratischen Gleichung lautet:
    \[
    b = \frac{1 \pm \sqrt{5}}{2}.
    \]
    Da $b_n > 0$ für alle $n$, folgt:
    \[
    b = \frac{1 + \sqrt{5}}{2}.
    \]

    \item \textbf{Beschränktheit der Folge:} Wir zeigen, dass die Folge $b_n$ beschränkt ist:
    \begin{itemize}
        \item Der Startwert ist $b_0 = 1$, und aus der Rekursion folgt:
        \[
        b_1 = 1 + \frac{1}{b_0} = 2, \quad b_2 = 1 + \frac{1}{b_1} = 1.5, \quad b_3 = 1 + \frac{1}{b_2} \approx 1.6667.
        \]
        \item Für $b_n > 0$ gilt aus der Definition:
        \[
        b_{n+1} = 1 + \frac{1}{b_n}.
        \]
        Da \( \frac{1}{b_n} > 0 \), folgt \( b_{n+1} > 1 \) für alle \( n \).
        \item Weiterhin zeigt die Rekursion, dass \( b_{n+1} \leq 2 \), da:
        \[
        b_{n+1} = 1 + \frac{1}{b_n} \leq 1 + 1 = 2.
        \]
        Somit liegt die Folge \( b_n \) im Intervall \( (1, 2] \) und ist beschränkt.

    \end{itemize}

    \item \textbf{Monotonie der Folge:} Wir zeigen, dass die Folge \( b_n \) monoton fallend ist:
    \begin{itemize}
        \item Aus der Rekursion \( b_{n+1} = 1 + \frac{1}{b_n} \) folgt, dass:
        \[
        b_{n+1} - b_n = 1 + \frac{1}{b_n} - b_n = \frac{1 - (b_n - 1)b_n}{b_n}.
        \]
        \item Da \( b_n > 1 \), ist \( b_n - 1 > 0 \). Weiterhin gilt:
        \[
        1 - (b_n - 1)b_n < 0 \quad \text{für alle } b_n > 1.
        \]
        Somit folgt \( b_{n+1} - b_n < 0 \), also ist die Folge monoton fallend.
    \end{itemize}

    \item \textbf{Konvergenz der Folge:} Da die Folge \( b_n \) beschränkt und monoton ist, konvergiert sie nach dem Monotonie-Kriterium. Der Grenzwert ist:
    \[
    b = \frac{1 + \sqrt{5}}{2}.
    \]
\end{enumerate}
\end{enumerate}

\section*{Aufgabe 4}
Gegeben seien zwei konvergente Reihen
\[
\sum_{n=1}^\infty a_n \quad \text{und} \quad \sum_{n=1}^\infty b_n
\]
sowie eine Konstante \( c \in \mathbb{R} \). Es ist zu zeigen, dass auch die Reihen
\[
\sum_{n=1}^\infty (a_n + b_n) \quad \text{und} \quad \sum_{n=1}^\infty (c \cdot a_n)
\]
konvergieren und es gelten:
\begin{enumerate}[label=(\alph*)]
    \item \[
        \sum_{n=1}^\infty (a_n + b_n) = \sum_{n=1}^\infty a_n + \sum_{n=1}^\infty b_n,
    \]
    \item \[
        \sum_{n=1}^\infty (c \cdot a_n) = c \cdot \sum_{n=1}^\infty a_n.
    \]
\end{enumerate}

\subsection*{Beweis}
\begin{enumerate}
    \item \textbf{Linearität der Addition:} Wir beweisen zunächst, dass die Summe der Reihe \(\sum_{n=1}^\infty (a_n + b_n)\) der Summe der Reihen \(\sum_{n=1}^\infty a_n\) und \(\sum_{n=1}^\infty b_n\) entspricht:
    \begin{enumerate}
        \item Da \(\sum_{n=1}^\infty a_n\) und \(\sum_{n=1}^\infty b_n\) konvergieren, existieren ihre jeweiligen Grenzwerte:
        \[
        S_a = \sum_{n=1}^\infty a_n, \quad S_b = \sum_{n=1}^\infty b_n.
        \]
        \item Die partielle Summe der Reihe \(\sum_{n=1}^\infty (a_n + b_n)\) ist definiert als:
        \[
        S_N = \sum_{n=1}^N (a_n + b_n) = \sum_{n=1}^N a_n + \sum_{n=1}^N b_n.
        \]
        \item Da die Grenzwerte existieren, folgt für \(N \to \infty\):
        \[
        \lim_{N \to \infty} S_N = \lim_{N \to \infty} \sum_{n=1}^N a_n + \lim_{N \to \infty} \sum_{n=1}^N b_n.
        \]
        Somit gilt:
        \[
        \sum_{n=1}^\infty (a_n + b_n) = \sum_{n=1}^\infty a_n + \sum_{n=1}^\infty b_n.
        \]
    \end{enumerate}

    \item \textbf{Skalarmultiplikation:} Wir beweisen, dass die Multiplikation der Reihe \(\sum_{n=1}^\infty (c \cdot a_n)\) mit einer Konstanten \( c \) dem Produkt von \( c \) mit der Summe der Reihe \(\sum_{n=1}^\infty a_n\) entspricht:
    \begin{enumerate}
        \item Da \(\sum_{n=1}^\infty a_n\) konvergiert, existiert der Grenzwert:
        \[
        S_a = \sum_{n=1}^\infty a_n.
        \]
        \item Die partielle Summe der Reihe \(\sum_{n=1}^\infty (c \cdot a_n)\) ist definiert als:
        \[
        S_N = \sum_{n=1}^N (c \cdot a_n) = c \cdot \sum_{n=1}^N a_n.
        \]
        \item Für \(N \to \infty\) folgt:
        \[
        \lim_{N \to \infty} S_N = c \cdot \lim_{N \to \infty} \sum_{n=1}^N a_n.
        \]
        Da \(\sum_{n=1}^\infty a_n\) konvergiert, gilt:
        \[
        \sum_{n=1}^\infty (c \cdot a_n) = c \cdot \sum_{n=1}^\infty a_n.
        \]
    \end{enumerate}
\end{enumerate}

\subsection*{Schlussfolgerung}
Die Reihen \(\sum_{n=1}^\infty (a_n + b_n)\) und \(\sum_{n=1}^\infty (c \cdot a_n)\) sind konvergent, wenn \(\sum_{n=1}^\infty a_n\) und \(\sum_{n=1}^\infty b_n\) konvergieren. Die Grenzwerte ergeben sich durch die Linearität der Addition und die Skalierung:
\[
\sum_{n=1}^\infty (a_n + b_n) = \sum_{n=1}^\infty a_n + \sum_{n=1}^\infty b_n,
\]
\[
\sum_{n=1}^\infty (c \cdot a_n) = c \cdot \sum_{n=1}^\infty a_n.
\]

\section*{Aufgabe 4}
Gegeben seien zwei konvergente Reihen
\[
\sum_{n=1}^\infty a_n \quad \text{und} \quad \sum_{n=1}^\infty b_n
\]
sowie eine Konstante \( c \in \mathbb{R} \). Es ist zu zeigen, dass auch die Reihen
\[
\sum_{n=1}^\infty (a_n + b_n) \quad \text{und} \quad \sum_{n=1}^\infty (c \cdot a_n)
\]
konvergieren und es gelten:
\begin{enumerate}[label=(\alph*)]
    \item \[
        \sum_{n=1}^\infty (a_n + b_n) = \sum_{n=1}^\infty a_n + \sum_{n=1}^\infty b_n,
    \]
    \item \[
        \sum_{n=1}^\infty (c \cdot a_n) = c \cdot \sum_{n=1}^\infty a_n.
    \]
\end{enumerate}

\subsection*{Beweis}
\begin{enumerate}
    \item \textbf{Linearität der Addition:} Wir beweisen zunächst, dass die Summe der Reihe \(\sum_{n=1}^\infty (a_n + b_n)\) der Summe der Reihen \(\sum_{n=1}^\infty a_n\) und \(\sum_{n=1}^\infty b_n\) entspricht:
    \begin{enumerate}
        \item Da \(\sum_{n=1}^\infty a_n\) und \(\sum_{n=1}^\infty b_n\) konvergieren, existieren ihre jeweiligen Grenzwerte:
        \[
        S_a = \sum_{n=1}^\infty a_n, \quad S_b = \sum_{n=1}^\infty b_n.
        \]
        Weiterhin ist bekannt, dass die Summe von zwei konvergenten Reihen ebenfalls konvergiert.
        \item Die partielle Summe der Reihe \(\sum_{n=1}^\infty (a_n + b_n)\) ist definiert als:
        \[
        S_N = \sum_{n=1}^N (a_n + b_n) = \sum_{n=1}^N a_n + \sum_{n=1}^N b_n.
        \]
        \item Da die Grenzwerte existieren und die Grenzwertbildung linear ist, folgt für \(N \to \infty\):
        \[
        \lim_{N \to \infty} S_N = \lim_{N \to \infty} \sum_{n=1}^N a_n + \lim_{N \to \infty} \sum_{n=1}^N b_n.
        \]
        Somit gilt:
        \[
        \sum_{n=1}^\infty (a_n + b_n) = \sum_{n=1}^\infty a_n + \sum_{n=1}^\infty b_n.
        \]
        Hierbei wurde die Stetigkeit der Addition verwendet.
    \end{enumerate}

    \item \textbf{Skalarmultiplikation:} Wir beweisen, dass die Multiplikation der Reihe \(\sum_{n=1}^\infty (c \cdot a_n)\) mit einer Konstanten \( c \) dem Produkt von \( c \) mit der Summe der Reihe \(\sum_{n=1}^\infty a_n\) entspricht:
    \begin{enumerate}
        \item Da \(\sum_{n=1}^\infty a_n\) konvergiert, existiert der Grenzwert:
        \[
        S_a = \sum_{n=1}^\infty a_n.
        \]
        Weiterhin bleibt die Multiplikation einer konvergenten Reihe mit einer Konstanten \( c \) ebenfalls konvergent, da die Multiplikation mit \( c \) keine Divergenz verursachen kann.
        \item Die partielle Summe der Reihe \(\sum_{n=1}^\infty (c \cdot a_n)\) ist definiert als:
        \[
        S_N = \sum_{n=1}^N (c \cdot a_n) = c \cdot \sum_{n=1}^N a_n.
        \]
        \item Für \(N \to \infty\) folgt:
        \[
        \lim_{N \to \infty} S_N = c \cdot \lim_{N \to \infty} \sum_{n=1}^N a_n.
        \]
        Da \(\sum_{n=1}^\infty a_n\) konvergiert, gilt:
        \[
        \sum_{n=1}^\infty (c \cdot a_n) = c \cdot \sum_{n=1}^\infty a_n.
        \]
        Hierbei wurde die Stetigkeit der Multiplikation verwendet.
    \end{enumerate}
\end{enumerate}

\subsection*{Schlussfolgerung}
Die Reihen \(\sum_{n=1}^\infty (a_n + b_n)\) und \(\sum_{n=1}^\infty (c \cdot a_n)\) sind konvergent, wenn \(\sum_{n=1}^\infty a_n\) und \(\sum_{n=1}^\infty b_n\) konvergieren. Die Grenzwerte ergeben sich durch die Linearität der Addition und die Skalierung:
\[
\sum_{n=1}^\infty (a_n + b_n) = \sum_{n=1}^\infty a_n + \sum_{n=1}^\infty b_n,
\]
\[
\sum_{n=1}^\infty (c \cdot a_n) = c \cdot \sum_{n=1}^\infty a_n.
\]
Hierbei wurden die Stetigkeit der Grenzwertoperationen und die Linearität der Addition und Multiplikation zentral verwendet.



\end{document}
