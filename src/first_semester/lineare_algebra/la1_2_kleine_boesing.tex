%! Author = felix
%! Date = 11.10.2024

% Preamble
\documentclass[11pt]{article}

% Packages
\usepackage{amsmath}
\usepackage{amsfonts}

% Document
\begin{document}

\title{Übungsblatt 2}
\author{Felix Kleine Bösing}
\maketitle

\subsection{Aufgabe 1.1}
Die Gerade \( G \) im \( \mathbb{R}^3 \) ist definiert durch die beiden Ebenengleichungen:
\[
2x_1 - 3x_2 + x_3 = 3
\]
\[
x_1 + x_2 + 2x_3 = 1
\]

Die Ebene \( E_a \) ist gegeben durch:
\[
5x_1 - 5x_2 + a^2 x_3 = 3a + 1
\]

\section*{Schritt 1: Bestimme den Richtungsvektor der Geraden \( G \)}

Die Gerade \( G \) ist der Schnitt der beiden Ebenen, deren Normalenvektoren wir bestimmen. Die Normalenvektoren der Ebenen lauten:

\[
\vec{n}_1 = (2, -3, 1), \quad \vec{n}_2 = (1, 1, 2)
\]

Der Richtungsvektor der Geraden \( G \) ergibt sich aus dem Kreuzprodukt der beiden Normalenvektoren:

\[
\vec{r}_G = \vec{n}_1 \times \vec{n}_2
\]
Berechne das Kreuzprodukt:
\[
\vec{r}_G = \begin{pmatrix}
-3 \times 2 - 1 \times 1 \\
1 \times 1 - 2 \times 2 \\
2 \times 1 - (-3) \times 1
\end{pmatrix}
=
\begin{pmatrix}
-7 \\
-3 \\
5
\end{pmatrix}
\]

Der Richtungsvektor der Geraden \( G \) ist also:
\[
\vec{r}_G = (-7, -3, 5)
\]

\section*{Schritt 2: Bestimme den Normalenvektor der Ebene \( E_a \)}

Die Ebene \( E_a \) ist durch die Gleichung:
\[
5x_1 - 5x_2 + a^2 x_3 = 3a + 1
\]
gegeben. Der Normalenvektor dieser Ebene ist:
\[
\vec{n}_E = (5, -5, a^2)
\]

\section*{Schritt 3: Prüfen der Parallelität}

Um zu überprüfen, ob die Gerade \( G \) parallel zur Ebene \( E_a \) ist, prüfen wir, ob der Richtungsvektor \( \vec{r}_G \) senkrecht zum Normalenvektor \( \vec{n}_E \) der Ebene ist.

Dazu berechnen wir das Skalarprodukt:
\[
\vec{r}_G \cdot \vec{n}_E = (-7) \cdot 5 + (-3) \cdot (-5) + 5 \cdot a^2
\]
\[
= -35 + 15 + 5a^2 = -20 + 5a^2
\]

Das Skalarprodukt muss 0 ergeben, wenn die Gerade parallel zur Ebene ist. Das bedeutet:
\[
-20 + 5a^2 = 0
\]
\[
a^2 = 4
\]
\[
a = \pm 2
\]

Die Gerade \( G \) ist also für \( a = 2 \) oder \( a = -2 \) parallel zur Ebene \( E_a \).

\section*{Schritt 4: Prüfen, wann die Gerade \( G \) in der Ebene liegt}

Damit die Gerade \( G \) in der Ebene liegt, muss sie einen Punkt gemeinsam mit der Ebene haben. Dazu setzen wir einen Punkt der Geraden in die Ebenengleichung von \( E_a \) ein und prüfen, ob die Gleichung für diesen Punkt erfüllt ist.

Die Parametergleichung der Geraden \( G \) kann aufgestellt werden, indem wir die beiden Ebenengleichungen lösen. Dieser Schritt ist nicht im Detail ausgeführt, aber wir können numerisch überprüfen, ob es eine Lösung gibt.

\subsection{Aufgabe 1.2}
Bestimmen Sie drei reelle Zahlen $a \in \mathbb{R}$, sodass das folgende lineare Gleichungssystem über $\mathbb{R}$ eindeutig, mehrdeutig bzw. überhaupt nicht lösbar ist:

\[
\begin{aligned}
    ax + y & = 1 \\
    4x + ay & = 2
\end{aligned}
\]

Über die Determinante der Koeffizientenmatrix des LGS lässt sich die Eindeutigkeit der Lösung bestimmen. Ist die
Determinante 0 gibt es keine eindeutige Lösung. Ist die Determinante ungleich 0 gibt es eine eindeutige Lösung.

\[
\det\begin{pmatrix}
    a & 1 \\
    4 & a
\end{pmatrix} = a \cdot a - 4 \cdot 1 = a^2 - 4
\]

\[
    a^2 - 4 = 0 \Rightarrow a^2 = 4 \Rightarrow a = \pm 2.
\]

Damit ist das LGS nicht eindeutig lösbar für $a = 2$ und $a = -2$. Setzt man nun $a = 2$ ein erhält man das folgende LGS:

\[
\begin{aligned}
    2x + y = 1 \\
    4x + 2y = 2
\end{aligned}
\]

Durch Multiplikation mit 2 der ersten Gleichung und anschließender Subtraktion der zweiten Gleichung erhält man:

\[
    0x + 0y = 0 \Rightarrow 0 = 0
\]

Da dies eine wahre Aussage ist, folgt daraus, dass das LGS unendlich viele Lösungen hat.
Geometrisch betrachtet handelt es sich um die gleiche Gerade, wodurch es unendlich viele "Schnittpunkte" gibt.

Für a = -2 ergibt sich folgende Gleichung für das LGS:

\[
\begin{aligned}
    -2x + y = 1 \\
    4x-2y = 2
\end{aligned}
\]

Durch Multiplikation mit 2 und Subtraktion der ersten Gleichung von der zweiten erhält man:
\[
    0 = 4
\]

Da dies eine falsche Aussage ist, folgt daraus, dass das LGS keine Lösung hat.
Geometrisch betrachtet handelt es sich um parallele Geraden, die sich nicht schneiden.
Daraus folgt, dass für jedes $a \in \mathbb{R}$, $a \neq -2$ und $a \neq 2$, das LGS eindeutig lösbar ist, für $a = 2$ unendlich viele Lösungen hat und für $a = -2$ keine Lösung hat.


\subsection{Aufgabe 1.3}

(a) Geometrische Bedeutung der möglichen Lösungen

\[
\begin{aligned}
    a_1x + b_1y + c_1z & = d_1 \\
    a_2x + b_2y + c_2z & = d_2 \\
    a_3x + b_3y + c_3z & = d_3
\end{aligned}
\]
mit Koeffizienten $a_i, b_i, c_i, d_i \in \mathbb{R}, i = 1, 2, 3$.
\bigskip

Wir betrachten hier einen Raum mit drei Dimensionen, in dem die Variablen x, y und z die Achsen bilden.
Dementsprechend handelt es sich bei jeder der drei Gleichungen um eine Ebene im Raum.

\begin{enumerate}
    \item  Keine Lösung: Die Ebenen liegen parallel zueinander im Raum
    \item  Eine Lösung: Die Ebenen schneiden sich in genau einem Punkt
    \item  Unendlich viele Lösungen: Mindestens zwei der Ebenen sind identisch oder parallel zueinander. Dies führt dazu, dass mindestens eine Gerade als Schnittmenge der Ebenen existiert und somit unendlich viele Schnittpunkte existieren.
\end{enumerate}
\bigskip

(b) Lösen des folgenden Gleichungssystems

\[
\begin{aligned}
    6x + 5y + 3z & = 1 \\
    x + 2y + z & = 4 \\
    2x - 2y - 2z & = 8
\end{aligned}
\]
\bigskip

Um das Gleichungssystem zu lösen, wenden wir den Gauß-Algorithmus an.

\[
\begin{pmatrix}
    6 & 5 & 3 & | & 1 \\
    1 & 2 & 1 & | & 4 \\
    2 & -2 & -2 & | & 8
\end{pmatrix}
\]

Multiplikaion der ersten Zeile mit $\frac{1}{6}$ und Subtraktion der ersten Zeile von der zweiten Zeile:

\[
\begin{pmatrix}
    6 & 5 & 3 & | & 1 \\
    0 & \frac{7}{6} & \frac{1}{2} & | & \frac{23}{6} \\
    2 & -2 & -2 & | & 8
\end{pmatrix}
\]

Multiplikation der ersten Zeile mit $\frac{1}{3}$ und Subtraktion der ersten Zeile von der dritten Zeile:

\[
\begin{pmatrix}
    6 & 5 & 3 & | & 1 \\
    0 & \frac{7}{6} & \frac{1}{2} & | & \frac{23}{6} \\
    0 & -\frac{11}{3} & -3 & | & \frac{23}{3}
\end{pmatrix}
\]

Multiplikation der zweiten Zeile mit $-\frac{22}{7}$ und Subtraktion der zweiten Zeile von der dritten Zeile:

\[
\begin{pmatrix}
    6 & 5 & 3 & | & 1 \\
    0 & \frac{7}{6} & \frac{1}{2} & | & \frac{23}{6} \\
    0 & 0 & -\frac{10}{7} & | & \frac{138}{7}
\end{pmatrix}
\]

Daraus folgt, dass:

\[
-\frac{10}{7}z = \frac{138}{70} \Rightarrow z= -\frac{69}{5}.
\]

Durch Einsetzen von z in die zweite Gleichung erhält man:

\[
\frac{7}{6}y = \frac{23}{6} - \frac{1}{2} \cdot \left(-\frac{69}{5}\right) \Rightarrow y = \frac{161}{15}
\]

Durch Einsetzen von y und z in die erste Gleichung erhält man:

\[
6x = 1 - 5 \cdot \frac{46}{5} - 3 \cdot -\frac{69}{5} \Rightarrow x = -\frac{3}{5}
\]

Ergebnis: Die drei Ebenen schneiden sich in genau einem Punkt der durch diesen Vektor beschrieben ist:

\[
\begin{pmatrix}
    -\frac{3}{5} \\
    \frac{46}{5} \\
    -\frac{69}{5}
\end{pmatrix}
\]

\subsection{Aufgabe 1.4}

Es soll gezeigt werden, welche von ${(0,0,0)}$ verschiedenen Lösungsmengen ein Gleichungssystem
unter Berücksichtigung zweier Bedingungen haben kann. Zunächst berechnen wir hierfür die Determinante der Koeffizientenmatrix des Gleichungssystems um daraus eine
Aussage über die Lösbarkeit des Gleichungssystem abzuleiten.

\[
\det\begin{pmatrix}
    1 & a & b \\
    b & 1 & a \\
    a & b & 1
\end{pmatrix} = 1 + a^3 + b^3 - 3ab
\]

Unter der ersten Bedingung $a = b = 1$ ergibt sich folgende Aussage:

\[
1 + 1^3 + 1^3 - 3 \cdot 1 \cdot 1 = 0
\]

Daraus folgt, dass es keine eindeutige Lösung gibt.
Betrachtet man das Gleichungssystem unter dieser Bedingung, so ergibt sich für alle drei Gleichungen:

\[
    x + y + z = 0
\]

Hinsichtlich der Lösungsmenge bedeutet dies, dass es unendlich viele Lösungen gibt, da es sich jeweils um dieselbe Ebene handelt.
\bigskip
Unter der zweiten Bedingung $a + b + 1 = 0$ $\Leftrightarrow$ $a= -b-1$ ergibt sich folgende Aussage:

\[
    1 + (-b-1)^3 + b^3 - 3b \cdot (-b-1) = b^3 - b^3 -3b^2 -3b - 1 + 3b^2 + 3b + 1 = 0
\]

Auch unter dieser Bedingung gibt es keine eindeutige Lösung, sodass es entweder keine oder unendlich viele Lösungen gibt.
Da die Ebenen weder identisch noch parallel zueinander sind, gibt es per ausschlussverfahren unendlich viele
Lösungen im Sinne von Schnittgeraden zwischen den Ebenen.

\bibliography{main}
\bibliographystyle{plain}

\end{document}
