%! Author = felix
%! Date = 11.10.2024

% Preamble
\documentclass[11pt]{article}

% Packages
\usepackage{amsmath}
\usepackage{amsfonts}

% Document
\begin{document}

\title{Übungsblatt 2}
\author{Felix Kleine Bösing}
\maketitle

\section*{Aufgabe 1}
Die Gerade \( G \) ist durch den Schnitt der beiden Ebenen definiert:

\[
2x_1 - 3x_2 + x_3 = 3
\]
\[
x_1 + x_2 + 2x_3 = 1
\]

Die Ebene \( E_a \) ist gegeben durch:
\[
5x_1 - 5x_2 + a^2 x_3 = 3a + 1
\]

Wir setzen \( x_3 = t \) als Parameter und lösen das Gleichungssystem nach \( x_1 \) und \( x_2 \) auf:

\[
2x_1 - 3x_2 + t = 3
\]
\[
x_1 + x_2 + 2t = 1
\]

Löse die zweite Gleichung nach \( x_1 \) auf:
\[
x_1 = 1 - x_2 - 2t
\]

Setze dies in die erste Gleichung ein:
\[
2(1 - x_2 - 2t) - 3x_2 + t = 3
\]
\[
2 - 2x_2 - 4t - 3x_2 + t = 3
\]
\[
-5x_2 - 3t = 1
\]
\[
x_2 = \frac{-3t - 1}{5}
\]

Setze \( x_2 \) in \( x_1 = 1 - x_2 - 2t \) ein:
\[
x_1 = 1 - \left(\frac{-3t - 1}{5}\right) - 2t
\]
\[
x_1 = 1 + \frac{3t + 1}{5} - 2t
\]
\[
x_1 = \frac{6 - 7t}{5}
\]

Die Parametergleichung der Geraden \( G \) lautet damit:
\[
(x_1, x_2, x_3) = \left( \frac{6 - 7t}{5}, \frac{-3t - 1}{5}, t \right)
\]

Durch Einsetzen der Parametergleichung der Geraden \( G \) in die Gleichung der Ebene \( E_a \) erhalten wir:
\[
5 \cdot \frac{6 - 7t}{5} - 5 \cdot \frac{-3t - 1}{5} + a^2 \cdot t = 3a + 1
\]
Vereinfacht ergibt sich:
\[
(6 - 7t) - (-3t - 1) + a^2 t = 3a + 1
\]
\[
7 - 4t + a^2 t = 3a + 1
\]
\[
(a^2-4) t = 3(a - 2)
\]

Wir suchen Werte für \( a \), bei denen die Gerade \( G \) die Ebene \( E_a \) schneidet, dazu parallel ist oder sich vollständig in der Ebene befindet.

In unserer Gleichung wir t eliminiert, wenn \(a = 2 \lor a = -2\)
Für alle \(a \neq 2 \land a \neq -2\) kann ein t bestimmt werden, sodass die Gleichung erfüllt ist.

\textbf{Fall 1: \( a = 2 \)}

Für \( a = 2 \) wird die Gleichung zu:
\[
7 - 4t + 4t = 6 + 1
\]
\[
7 = 7
\]
Dies ist immer wahr, unabhängig von \( t \). Das bedeutet, dass jeder Punkt auf der Geraden die Gleichung der Ebene erfüllt. **Die Gerade liegt vollständig in der Ebene**, wenn \( a = 2 \).

\textbf{Fall 2: \( a = -2 \)}

Für \( a = -2 \) wird die Gleichung zu:
\[
7 - 4t + 4t = -6 + 1
\]
\[
7 = -5
\]
Dies ist ein Widerspruch. Das bedeutet, dass kein Punkt auf der Geraden die Gleichung der Ebene erfüllt. **Die Gerade ist parallel zur Ebene**, wenn \( a = -2 \).

\textbf{Fall 3: \( a \neq 2 \land a \neq -2 \)}

In diesem Fall können wir \( t \) berechnen:
\[
( -4 + a^2 )t = 3a - 6
\]
\[
t = \frac{3a - 6}{a^2 - 4}
\]
Es gibt genau einen Wert für \( t \), bei dem die Gerade die Ebene schneidet.

Zusammenfassend:
\begin{enumerate}
  \item Für \( a = 2 \) liegt die Gerade vollständig in der Ebene.
  \item Für \( a = -2 \) ist die Gerade parallel zur Ebene.
  \item Für \( a \neq 2 \) und \( a \neq -2 \) schneidet die Gerade die Ebene in genau einem Punkt.
\end{enumerate}
\section*{Aufgabe 2}


Gegeben ist das folgende lineare Gleichungssystem:
\[
\begin{aligned}
a x + (a+1) y + (a+2) z &= 0 \\
(a+3) x + (a+4) y + (a+5) z &= 0 \\
(a+6) x + (a+7) y + (a+8) z &= 0
\end{aligned}
\]

Wir sollen zeigen, dass das Gleichungssystem nicht nur die triviale Lösung besitzt.


Die triviale Lösung ist \( x = y = z = 0 \). Um zu zeigen, dass es noch andere Lösungen gibt, bestimmen wir zunächst die Determinante der Koeffizientenmatrix.

\[
\det\begin{pmatrix}
    a & a+1 & a+2 \\
    a +3 & a+4 & a+5 \\
    a+6 & a+7 & a+8
\end{pmatrix} = 0
\]

Da die Determinante der Koeffizientenmatrix gleich 0 ist, können wir davon ausgehen, dass das LGS keine oder unendlich Lösungen hat.
Weitergehend bestimmen wir den Rang um festzustellen, ob es keine oder unendlich viele Lösungen gibt.

\begin{enumerate}
  \item Subtraktion der ersten Zeile von der zweiten und dritten Zeile:
  \[
  \begin{pmatrix}
  a & a+1 & a+2 \\
  3 & 3 & 3 \\
  6 & 6 & 6
  \end{pmatrix}
  \]
  \item Subtraktion den Doppelten der zweiten Zeile von der dritten Zeile:
  \[
  \begin{pmatrix}
  a & a+1 & a+2 \\
  3 & 3 & 3 \\
  0 & 0 & 0
  \end{pmatrix}
  \]
\end{enumerate}

Der Rang dieser Matrix ist 2, da zwei Zeilen nicht nur aus Nullen bestehen. Das bedeutet, dass es unendlich viele Lösungen gibt.

Um nun die Lösungen zu bestimmen, lösen wir das zuvor vereinfachte Gleichungssystem:

\[
\begin{aligned}
a x + (a+1) y + (a+2) z = 0 \\
3x + 3y + 3z = 0 \\
x + y + z = 0
\end{aligned}
\]

Lösen der dritten Gleichung nach \( y \):
\[
y = -x - z
\]

Einsetzen in die erste Gleichung:
\[
a x + (a+1) (-x - z) + (a+2) z = 0
\]

Vereinfachen:
\[
a x - (a+1) x - (a+1) z + (a+2) z = 0
\]
\[
(a - (a+1)) x + ((a+2) - (a+1)) z = 0
\]
\[
-1x + z = 0 \quad \Rightarrow \quad x = z
\]

Da \( y = -x - z \), ergibt sich:
\[
y = -z - z = -2z
\]

Zusammenfassend:
Das Gleichungssystem hat nicht nur die triviale Lösung. Die allgemeine Lösung ist:
\[
x = z, \quad y = -2z
\]

Damit ist die Lösung des Gleichungssystems:
\[
(x, y, z) = (z, -2z, z) \quad \text{für beliebiges} \ z \in \mathbb{R}
\]
Das Gleichungssystem besitzt also unendlich viele Lösungen.

\section*{Aufgabe 3}

Gegeben seien Mengen \( X, Y, Z \) sowie zwei Abbildungen \( f : X \to Y \) und \( g : Y \to Z \). Wir untersuchen nun die gegebenen Aussagen:

\medskip

\textbf{(a)} Sind \( f \) und \( g \) injektiv (bzw. surjektiv), so ist \( g \circ f \) injektiv (bzw. surjektiv).
\smallskip

\textbf{Injektivität:} Wenn \( f \) und \( g \) beide injektiv sind, bedeutet dies, dass \( f \) verschiedene Elemente von \( X \) auf verschiedene Elemente in \( Y \) abbildet und \( g \) verschiedene Elemente von \( Y \) auf verschiedene Elemente in \( Z \). Da beide Funktionen injektiv sind, muss auch die Komposition \( g \circ f \) injektiv sein.

Beweis:
\[
g(f(x_1)) = g(f(x_2)) \implies f(x_1) = f(x_2) \quad \text{(weil \( g \) injektiv ist)}
\]
\[
f(x_1) = f(x_2) \implies x_1 = x_2 \quad \text{(weil \( f \) injektiv ist)}
\]
Also ist \( g \circ f \) injektiv.

\medskip

\textbf{Surjektivität:} Wenn \( f \) und \( g \) beide surjektiv sind, dann bedeutet dies, dass jedes Element in \( Z \) durch \( g \) aus einem Element von \( Y \) erreicht wird und jedes Element in \( Y \) durch \( f \) aus einem Element von \( X \) erreicht wird. Also ist die Komposition \( g \circ f \) surjektiv, da die gesamte Zielmenge abgedeckt wird.

Beweis:
\[
\forall z \in Z, \exists y \in Y \ \text{mit} \ g(y) = z
\]
\[
\forall y \in Y, \exists x \in X \ \text{mit} \ f(x) = y
\]
Damit existiert für jedes \( z \in Z \) ein \( x \in X \), sodass \( g(f(x)) = z \). Also ist \( g \circ f \) surjektiv.

\bigskip

\textbf{(b)} Ist \( g \circ f \) injektiv (bzw. surjektiv), so ist \( f \) injektiv (bzw. \( g \) surjektiv).

\smallskip

\textbf{Injektivität:} Wenn \( g \circ f \) injektiv ist, folgt, dass auch \( f \) injektiv sein muss. Andernfalls gäbe es zwei verschiedene Werte \( x_1, x_2 \in X \), für die \( f(x_1) = f(x_2) \). Dies würde bedeuten, dass \( g(f(x_1)) = g(f(x_2)) \), was der Injektivität von \( g \circ f \) widerspricht. Daher muss \( f \) injektiv sein.

\medskip

\textbf{Surjektivität:} Wenn \( g \circ f \) surjektiv ist, muss auch \( g \) surjektiv sein. Denn wenn \( g \) nicht surjektiv wäre, könnte \( g \circ f \) nicht jedes Element in \( Z \) erreichen, was der Surjektivität von \( g \circ f \) widerspricht.

\bigskip

\textbf{(c)} Ist \( g \circ f \) injektiv und \( f \) surjektiv, so ist \( g \) injektiv.

\textit{Beweis:} Da \( f \) surjektiv ist, erreicht jedes Element von \( Y \) durch \( f \) ein Element von \( X \). Wenn \( g \circ f \) injektiv ist, muss auch \( g \) injektiv sein. Andernfalls gäbe es zwei verschiedene Elemente \( y_1, y_2 \in Y \), die auf dasselbe Element in \( Z \) abgebildet werden, was der Injektivität von \( g \circ f \) widerspricht.

\bigskip

\textbf{(d)} Beispiel: \( g \circ f \) ist bijektiv, aber \( f \) ist nicht surjektiv und \( g \) ist nicht injektiv.

Wir konstruieren ein Beispiel, bei dem \( g \circ f \) bijektiv ist, aber \( f \) nicht surjektiv und \( g \) nicht injektiv ist.

Seien:
\[
X = \{1, 2\}, \quad Y = \{1, 2, 3\}, \quad Z = \{1, 2\}
\]
Definiere die Funktionen \( f \) und \( g \) wie folgt:
\[
f(1) = 1, \quad f(2) = 2 \quad \text{(nicht surjektiv, da \( 3 \) in \( Y \) nicht erreicht wird)},
\]
\[
g(1) = 1, \quad g(2) = 2, \quad g(3) = 2 \quad \text{(nicht injektiv, da \( g(2) = g(3) \))}.
\]
Die Komposition \( g \circ f \) ergibt:
\[
g(f(1)) = g(1) = 1, \quad g(f(2)) = g(2) = 2
\]
Also ist \( g \circ f \) bijektiv, obwohl \( f \) nicht surjektiv und \( g \) nicht injektiv ist.


\section*{Aufgabe 4}

Wir sollen zeigen, dass die Abbildung
\[
f \mapsto (p_i \circ f)_{i \in I}
\]
der Menge aller Abbildungen \( \text{Abb}(X, \prod_{i \in I} X_i) \) in das kartesische Produkt \( \prod_{i \in I} \text{Abb}(X, X_i) \) bijektiv ist.

\medskip

Um dies zu zeigen, müssen wir zeigen, dass die Abbildung sowohl injektiv als auch surjektiv ist.

\textbf{1. Injektivität}

Wir zeigen zunächst, dass die Abbildung injektiv ist. Das bedeutet, wenn für zwei Abbildungen \( f_1, f_2 \in \text{Abb}(X, \prod_{i \in I} X_i) \) gilt, dass \( (p_i \circ f_1)(x) = (p_i \circ f_2)(x) \) für alle \( i \in I \) und für jedes \( x \in X \), dann folgt daraus \( f_1 = f_2 \).

Beweis:
Seien \( f_1, f_2 : X \to \prod_{i \in I} X_i \) zwei Abbildungen, sodass \( (p_i \circ f_1) = (p_i \circ f_2) \) für alle \( i \in I \). Das bedeutet, dass für jedes \( x \in X \) und für alle \( i \in I \) gilt:
\[
p_i(f_1(x)) = p_i(f_2(x))
\]
Da \( p_i \) die Projektion auf die \( i \)-te Komponente ist, folgt daraus, dass \( f_1(x) = f_2(x) \) für jedes \( x \in X \). Somit ist \( f_1 = f_2 \), also ist die Abbildung injektiv.

\textbf{2. Surjektivität}

Wir zeigen nun, dass die Abbildung surjektiv ist. Das bedeutet, dass für jede Familie von Abbildungen \( (g_i)_{i \in I} \), wobei \( g_i : X \to X_i \), eine Abbildung \( f \in \text{Abb}(X, \prod_{i \in I} X_i) \) existiert, sodass \( p_i \circ f = g_i \) für alle \( i \in I \).

Beweis:
Sei \( (g_i)_{i \in I} \) eine Familie von Abbildungen \( g_i : X \to X_i \). Definiere eine Abbildung \( f : X \to \prod_{i \in I} X_i \) durch:
\[
f(x) = (g_i(x))_{i \in I}
\]
Das heißt, \( f(x) \) ist das Tupel, dessen \( i \)-te Komponente \( g_i(x) \) ist. Für jedes \( i \in I \) gilt dann:
\[
p_i(f(x)) = g_i(x)
\]
Also ist \( p_i \circ f = g_i \), und damit ist die Abbildung surjektiv.


Zusammenfassend:
Da die Abbildung sowohl injektiv als auch surjektiv ist, ist sie per Defintion bijektiv.


\bibliography{main}
\bibliographystyle{plain}

\end{document}
