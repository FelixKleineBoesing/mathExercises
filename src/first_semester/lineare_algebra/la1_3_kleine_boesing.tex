%! Author = felix
%! Date = 11.10.2024

% Preamble
\documentclass[11pt]{article}

% Packages
\usepackage{amsmath}
\usepackage{amsfonts}

% Document
\begin{document}

\title{Übungsblatt 3}
\author{Felix Kleine Bösing}
\maketitle

\section*{Aufgabe 1}


Es sei das folgende lineare Gleichungssystem über dem Körper \( K \) gegeben:
\[
x + y + z = 0
\]
\[
x - y - z = 0
\]
\[
x + z = 0
\]
Bestimmen Sie die Lösungen dieses Gleichungssystems, falls:

\subsection*{(a) \( K = \mathbb{C} \)}

\textbf{Schritt 1:} Aus der dritten Gleichung folgt:
\[
x = -z
\]

\textbf{Schritt 2:} Einsetzen von \( x = -z \) in die anderen Gleichungen:
\[
-z + y + z = 0 \implies y = 0
\]
\[
-z - 0 - z = 0 \implies z = 0
\]

\textbf{Schritt 3:} Damit ist:
\[
x = 0, \quad y = 0, \quad z = 0
\]
Die Lösung über \( \mathbb{C} \) ist daher nur die triviale Lösung:
\[
(0, 0, 0)
\]

\subsection*{(b) \( K = \mathbb{F}_2 \)}

Da wir uns in \( \mathbb{F}_2 \) befinden, sind die einzigen möglichen Werte für die Variablen \( 0 \) und \( 1 \).
Anders gesagt müssen wir in \( \mathbb{F}_2 \) die Gleichungen mod 2 betrachten. Das bedeutet für die Multiplikation und Addition, dass:
\[
    1 + 1 = 0, \quad 1 \cdot 1 = 1
    -1 = 1
\]

\textbf{Schritt 1:} Aus der dritten Gleichung \( x + z = 0 \) folgt in \( \mathbb{F}_2 \), dass
\[
x = z
\]

\textbf{Schritt 2:} Setzen wir \( x = z \) in die erste Gleichung \( x + y + z = 0 \) ein:
\[
z + y + z = 0 \implies 2z + y = 0
\]
Da in \( \mathbb{F}_2 \) \( 2z = 0 \) ist, bleibt:
\[
y = 0
\]

\textbf{Schritt 3:} Nun wissen wir, dass \( y = 0 \) ist, und \( x = z \) gilt. Also gibt es zwei mögliche Fälle:
\begin{itemize}
    \item Wenn \( z = 0 \), dann ist \( x = 0 \). Die Lösung ist:
    \[
    (x, y, z) = (0, 0, 0)
    \]
    \item Wenn \( z = 1 \), dann ist \( x = 1 \). Die Lösung ist:
    \[
    (x, y, z) = (1, 0, 1)
    \]
\end{itemize}

\textbf{Zusammenfassung:} Über \( \mathbb{F}_2 \) gibt es zwei Lösungen:
\[
(0, 0, 0) \quad \text{und} \quad (1, 0, 1)
\]

\section*{Aufgabe 2}

Gesucht sind alle komplexen Zahlen \( z = x + iy \), die die Gleichung \( z^2 = z_0 \) erfüllen, wobei \( z_0 = a + ib \in \mathbb{C} \).

\textbf{Schritt 1:} Wir berechnen das Quadrat von \( z \):
\[
z^2 = (x + iy)^2 = x^2 + 2ixy + i^2 y^2 = x^2 - y^2 + 2ixy
\]
Dies muss gleich \( z_0 = a + ib \) sein, also ergeben sich die beiden Gleichungen:
\[
x^2 - y^2 = a \quad (1), \quad 2xy = b \quad (2)
\]

\textbf{Schritt 2:} Lösen der Gleichung \( 2xy = b \) nach \( y \):
\[
y = \frac{b}{2x} \quad \text{für} \quad x \neq 0
\]

\textbf{Schritt 3:} Einsetzen von \( y \) in die Gleichung \( x^2 - y^2 = a \):
\[
x^2 - \left( \frac{b}{2x} \right)^2 = a \implies x^2 - \frac{b^2}{4x^2} = a
\]
Multiplizieren mit \( 4x^2 \) ergibt:
\[
4x^4 - 4a x^2 - b^2 = 0
\]
Setze \( u = x^2 \), um die quadratische Gleichung zu lösen:
\[
4u^2 - 4a u - b^2 = 0
\]
Die Lösung ist:
\[
u = \frac{a \pm \sqrt{a^2 + b^2}}{2}
\]

\textbf{Schritt 4:} Bestimmen von \( x \) und \( y \):
\[
x = \pm \sqrt{\frac{a + \sqrt{a^2 + b^2}}{2}}, \quad y = \frac{b}{2x}
\]

\textbf{Lösungen:} Es gibt zwei Lösungen:
\[
z = \pm \left( \sqrt{\frac{a + \sqrt{a^2 + b^2}}{2}} + i \frac{b}{2 \sqrt{\frac{a + \sqrt{a^2 + b^2}}{2}}} \right)
\]

\section*{Aufgabe 3}

Es sei \( (G, \cdot) \) eine Gruppe. Wir zeigen, dass die Aussagen (a), (b) und (c) für eine nichtleere Teilmenge \( H \subseteq G \) äquivalent sind.

\textbf{Beweis: (a) impliziert (b)} \\
Angenommen, \( H \) erfüllt die Bedingungen von (a), d.h. für alle \( h_1, h_2 \in H \) gilt \( h_1^{-1} \in H \) und \( h_1 \cdot h_2 \in H \). \\
Nun zeigen wir, dass für alle \( h_1, h_2 \in H \) auch \( h_1 \cdot h_2^{-1} \in H \) gilt. Da nach (a) \( h_2^{-1} \in H \), gilt durch die Abgeschlossenheit der Gruppenoperation auch \( h_1 \cdot h_2^{-1} \in H \). Somit ist (b) erfüllt.

\textbf{Beweis: (b) impliziert (c)} \\
Angenommen, \( H \) erfüllt die Bedingung (b), d.h. für alle \( h_1, h_2 \in H \) gilt \( h_1 \cdot h_2^{-1} \in H \). \\
Da \( h_2 \in H \), gilt auch \( h_2^{-1} \in H \). Da \( H \) unter Inversenbildung abgeschlossen ist, ist \( H \) eine Gruppe. Somit gilt auch (c).

\textbf{Beweis: (c) impliziert (a)} \\
Angenommen, \( H \) ist eine Gruppe. Wir müssen zeigen, dass \( h_1^{-1} \in H \) und \( h_1 \cdot h_2 \in H \). Da \( H \) eine Gruppe ist, gilt per Definition die Abgeschlossenheit der Operation und die Inversenbildung, also ist (a) erfüllt.

\textbf{Fazit:} Die Aussagen (a), (b) und (c) sind äquivalent.

\bibliography{main}
\bibliographystyle{plain}

\end{document}
