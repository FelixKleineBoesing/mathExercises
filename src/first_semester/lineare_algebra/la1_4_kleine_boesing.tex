%! Author = felix
%! Date = 11.10.2024

% Preamble
\documentclass[11pt]{article}

% Packages
\usepackage{amsmath}
\usepackage{amsfonts}

% Document
\begin{document}

\title{Übungsblatt 4}
\author{Felix Kleine Bösing}
\maketitle

\section*{Aufgabe 1}

Untersuchen Sie, welche der folgenden Teilmengen Untervektorräume von \( \mathbb{Q}^3 \) sind:
\[
\begin{aligned}
    M_1 &= \{(x, y, z) \in \mathbb{Q}^3 : x, y, z \geq 0\}, \\
    M_2 &= \{(x, y, z) \in \mathbb{Q}^3 : 3x + y + z = 5\}, \\
    M_3 &= \{(x, y, z) \in \mathbb{Q}^3 : x + 2y = 3z\}, \\
    M_4 &= \{(x, y, z) \in \mathbb{Q}^3 : xy - z = 0\}.
\end{aligned}
\]

\subsection*{Teil (a)}

\textbf{Beweis:} Um zu überprüfen, ob \( M_1 \) ein Untervektorraum von \( \mathbb{Q}^3 \) ist, müssen wir die folgenden Eigenschaften zeigen:

\begin{enumerate}
    \item \textbf{Der Nullvektor muss enthalten sein:} Der Nullvektor in \( \mathbb{Q}^3 \) ist \((0, 0, 0)\). Da \(0 \geq 0\) für jede Komponente gilt, gehört der Nullvektor zu \( M_1 \).

    \item \textbf{Abgeschlossenheit unter Addition:} Nehmen wir an, dass \( (x_1, y_1, z_1), (x_2, y_2, z_2) \in M_1 \). Dann sind \( x_1, y_1, z_1 \geq 0 \) und \( x_2, y_2, z_2 \geq 0 \). Für die Summe \((x_1 + x_2, y_1 + y_2, z_1 + z_2)\) gilt ebenfalls \( x_1 + x_2 \geq 0 \), \( y_1 + y_2 \geq 0 \) und \( z_1 + z_2 \geq 0 \), sodass die Summe auch in \( M_1 \) liegt.

    \item \textbf{Abgeschlossenheit unter Skalarmultiplikation:} Sei \( (x, y, z) \in M_1 \) und \( c \in \mathbb{Q} \). Wenn \( c < 0 \), dann wird eine oder mehrere der Komponenten \( cx, cy, cz \) negativ, was die Bedingung \( x, y, z \geq 0 \) verletzt. Daher ist \( M_1 \) nicht unter Skalarmultiplikation abgeschlossen.
\end{enumerate}

Da \( M_1 \) nicht unter Skalarmultiplikation abgeschlossen ist, ist es \textbf{kein Untervektorraum} von \( \mathbb{Q}^3 \).

\subsection*{Teil (b)}

\textbf{Beweis:} Untersuchen wir, ob \( M_2 \) ein Untervektorraum von \( \mathbb{Q}^3 \) ist.

\begin{enumerate}
    \item \textbf{Der Nullvektor muss enthalten sein:} Der Nullvektor in \( \mathbb{Q}^3 \) ist \((0, 0, 0)\). Setzen wir diesen in die Bedingung \(3x + y + z = 5\) ein, so erhalten wir:
   \[
   3 \cdot 0 + 0 + 0 = 0 \neq 5.
   \]
   Daher gehört der Nullvektor \textbf{nicht} zu \( M_2 \).
\end{enumerate}

Da der Nullvektor nicht in \( M_2 \) liegt, ist \( M_2 \) \textbf{kein Untervektorraum} von \( \mathbb{Q}^3 \).

\subsection*{Teil (c)}

\textbf{Beweis:} Untersuchen wir, ob \( M_3 \) ein Untervektorraum von \( \mathbb{Q}^3 \) ist.

\begin{enumerate}
    \item \textbf{Der Nullvektor muss enthalten sein:} Der Nullvektor in \( \mathbb{Q}^3 \) ist \((0, 0, 0)\). Setzen wir diesen in die Bedingung \(x + 2y = 3z\) ein, so erhalten wir:
   \[
   0 + 2 \cdot 0 = 3 \cdot 0,
   \]
   was offensichtlich wahr ist. Daher gehört der Nullvektor zu \( M_3 \).

    \item \textbf{Abgeschlossenheit unter Addition:} Nehmen wir an, dass \( (x_1, y_1, z_1), (x_2, y_2, z_2) \in M_3 \). Dann gilt:
   \[
   x_1 + 2y_1 = 3z_1 \quad \text{und} \quad x_2 + 2y_2 = 3z_2.
   \]
   Für die Summe \((x_1 + x_2, y_1 + y_2, z_1 + z_2)\) ergibt sich:
   \[
   (x_1 + x_2) + 2(y_1 + y_2) = (x_1 + 2y_1) + (x_2 + 2y_2) = 3z_1 + 3z_2 = 3(z_1 + z_2),
   \]
   sodass die Summe ebenfalls die Bedingung erfüllt. \( M_3 \) ist also unter Addition abgeschlossen.

    \item \textbf{Abgeschlossenheit unter Skalarmultiplikation:} Sei \( (x, y, z) \in M_3 \) und \( c \in \mathbb{Q} \). Dann gilt:
   \[
   x + 2y = 3z.
   \]
   Für das Produkt \( c \cdot (x, y, z) = (cx, cy, cz) \) erhalten wir:
   \[
   cx + 2(cy) = c(x + 2y) = c \cdot 3z = 3(cz),
   \]
   was zeigt, dass auch \( (cx, cy, cz) \) die Bedingung erfüllt. Somit ist \( M_3 \) unter Skalarmultiplikation abgeschlossen.
\end{enumerate}

Da \( M_3 \) sowohl den Nullvektor enthält als auch unter Addition und Skalarmultiplikation abgeschlossen ist, ist \( M_3 \) ein \textbf{Untervektorraum} von \( \mathbb{Q}^3 \).

\subsection*{Teil (d)}

\textbf{Beweis:} Untersuchen wir, ob \( M_4 \) ein Untervektorraum von \( \mathbb{Q}^3 \) ist.

\begin{enumerate}
    \item \textbf{Der Nullvektor muss enthalten sein:} Der Nullvektor in \( \mathbb{Q}^3 \) ist \((0, 0, 0)\). Setzen wir diesen in die Bedingung \(xy - z = 0\) ein, so erhalten wir:
   \[
   0 \cdot 0 - 0 = 0,
   \]
   was offensichtlich wahr ist. Daher gehört der Nullvektor zu \( M_4 \).

    \item \textbf{Abgeschlossenheit unter Addition:} Nehmen wir an, dass \( (x_1, y_1, z_1), (x_2, y_2, z_2) \in M_4 \), also \( x_1 y_1 = z_1 \) und \( x_2 y_2 = z_2 \). Für die Summe \((x_1 + x_2, y_1 + y_2, z_1 + z_2)\) ergibt sich jedoch:
   \[
   (x_1 + x_2)(y_1 + y_2) = x_1 y_1 + x_1 y_2 + x_2 y_1 + x_2 y_2.
   \]
   Da zusätzliche Kreuzterme wie \( x_1 y_2 \) und \( x_2 y_1 \) auftreten, ist im Allgemeinen \((x_1 + x_2)(y_1 + y_2) \neq z_1 + z_2\). Somit ist \( M_4 \) nicht unter Addition abgeschlossen.
\end{enumerate}

Da \( M_4 \) nicht unter Addition abgeschlossen ist, ist es \textbf{kein Untervektorraum} von \( \mathbb{Q}^3 \).

\section*{Aufgabe 4.2}

\subsection*{Teil (a)}

\textbf{Beweis:} Wir sollen ein Beispiel eines Vektorraums \( V \) und einer Teilmenge \( M \subseteq V \) finden, sodass für alle \( v, w \in M \) mit \( v \neq w \) die Vektoren \( v \) und \( w \) linear unabhängig sind, die Menge \( M \) jedoch linear abhängig ist.

Ein solches Beispiel ist der Vektorraum \( V = \mathbb{R}^3 \) und die Teilmenge
\[
M = \left\{ \begin{pmatrix} 1 \\ 0 \\ 0 \end{pmatrix}, \begin{pmatrix} 0 \\ 1 \\ 0 \end{pmatrix}, \begin{pmatrix} 1 \\ 1 \\ 0 \end{pmatrix} \right\}.
\]

Betrachten wir die Eigenschaften dieser Vektoren:

\begin{enumerate}
    \item Für jedes Paar unterschiedlicher Vektoren \( v, w \in M \) sind \( v \) und \( w \) linear unabhängig. Zum Beispiel sind \( \begin{pmatrix} 1 \\ 0 \\ 0 \end{pmatrix} \) und \( \begin{pmatrix} 0 \\ 1 \\ 0 \end{pmatrix} \) linear unabhängig, da keine Linearkombination \( c_1 \begin{pmatrix} 1 \\ 0 \\ 0 \end{pmatrix} + c_2 \begin{pmatrix} 0 \\ 1 \\ 0 \end{pmatrix} = 0 \) für \( c_1, c_2 \in \mathbb{R} \) außer \( c_1 = c_2 = 0 \) existiert.

    \item Die Menge \( M \) ist jedoch linear abhängig, da
    \[
    \begin{pmatrix} 1 \\ 0 \\ 0 \end{pmatrix} + \begin{pmatrix} 0 \\ 1 \\ 0 \end{pmatrix} - \begin{pmatrix} 1 \\ 1 \\ 0 \end{pmatrix} = \begin{pmatrix} 0 \\ 0 \\ 0 \end{pmatrix}.
    \]
    Dies zeigt, dass die Vektoren in \( M \) eine lineare Abhängigkeit aufweisen.
\end{enumerate}

Somit erfüllt die Teilmenge \( M \) die Bedingungen der Aufgabe.

\subsection*{Teil (b)}

\textbf{Beweis:} Gegeben sei ein Körper \( K \) und ein \( K \)-Vektorraum \( V \). Sei \( M = \{v_1, \dots, v_n\} \subseteq V \) eine Teilmenge, wobei \( 0 \notin M \). Wir sollen zeigen, dass \( M \) genau dann linear unabhängig ist, wenn für alle \( i \in \{1, 2, \dots, n-1\} \) gilt:
\[
\langle v_1, \dots, v_i \rangle \cap \langle v_{i+1}, \dots, v_n \rangle = \{0\}.
\]

\begin{enumerate}
    \item \textbf{Notwendigkeit:} Angenommen, \( M \) ist linear unabhängig. Dann bedeutet dies, dass keine nicht-triviale Linearkombination der Vektoren in \( M \) den Nullvektor ergibt. Insbesondere ist jeder Vektor \( v_i \) nicht in der Linearkombination der anderen Vektoren, was impliziert, dass für jedes \( i \) die Schnittmenge \( \langle v_1, \dots, v_i \rangle \cap \langle v_{i+1}, \dots, v_n \rangle \) nur den Nullvektor enthält, also
    \[
    \langle v_1, \dots, v_i \rangle \cap \langle v_{i+1}, \dots, v_n \rangle = \{0\}.
    \]

    \item \textbf{Hinreichend:} Angenommen, für alle \( i \in \{1, 2, \dots, n-1\} \) gilt \( \langle v_1, \dots, v_i \rangle \cap \langle v_{i+1}, \dots, v_n \rangle = \{0\} \). Dies bedeutet, dass es keine nicht-triviale Linearkombination von \( v_1, \dots, v_i \) gibt, die auch als Linearkombination von \( v_{i+1}, \dots, v_n \) ausgedrückt werden kann. Folglich ist \( M \) linear unabhängig, da jede Linearkombination, die den Nullvektor ergibt, nur die triviale Lösung \( c_1 = c_2 = \dots = c_n = 0 \) hat.
\end{enumerate}

Damit ist gezeigt, dass \( M \) genau dann linear unabhängig ist, wenn für alle \( i \in \{1, 2, \dots, n-1\} \) gilt:
\[
\langle v_1, \dots, v_i \rangle \cap \langle v_{i+1}, \dots, v_n \rangle = \{0\}.
\]


\section*{Aufgabe 4.3}

Gegeben seien ein Körper \( K \) und die Untervektorräume \( U, V \subseteq K^n \). Definieren wir die folgenden Mengen:
\[
\begin{aligned}
    U \cap V &= \{ x \in K^n : x \in U \text{ und } x \in V \}, \\
    U \cup V &= \{ x \in K^n : x \in U \text{ oder } x \in V \}, \\
    U + V &= \{ x \in K^n : \text{es existieren } u \in U \text{ und } v \in V \text{ mit } x = u + v \}.
\end{aligned}
\]

Zeigen Sie:

\subsection*{Teil (a)}

\textbf{Beweis:} Um zu zeigen, dass die Mengen \( U \cap V \) und \( U + V \) Untervektorräume des \( K^n \) sind, müssen wir überprüfen, ob sie die Bedingungen für einen Untervektorraum erfüllen:

\begin{enumerate}
    \item Die Menge muss den Nullvektor enthalten.
    \item Sie muss unter Addition abgeschlossen sein.
    \item Sie muss unter Skalarmultiplikation abgeschlossen sein.
\end{enumerate}

\begin{enumerate}
    \item \textbf{Untervektorraum \( U \cap V \):}
    \begin{enumerate}
        \item \textbf{Nullvektor:} Da \( U \) und \( V \) Untervektorräume von \( K^n \) sind, enthalten beide den Nullvektor \( 0 \). Da \( 0 \in U \) und \( 0 \in V \) gilt, folgt \( 0 \in U \cap V \).

        \item \textbf{Abgeschlossenheit unter Addition:} Sei \( x, y \in U \cap V \). Dann gilt \( x \in U \), \( x \in V \), \( y \in U \) und \( y \in V \). Da \( U \) und \( V \) jeweils unter Addition abgeschlossen sind, ist auch \( x + y \in U \) und \( x + y \in V \). Daher gilt \( x + y \in U \cap V \), und \( U \cap V \) ist unter Addition abgeschlossen.

        \item \textbf{Abgeschlossenheit unter Skalarmultiplikation:} Sei \( x \in U \cap V \) und \( c \in K \). Da \( x \in U \) und \( x \in V \) sowie \( U \) und \( V \) jeweils unter Skalarmultiplikation abgeschlossen sind, folgt \( c \cdot x \in U \) und \( c \cdot x \in V \). Somit ist \( c \cdot x \in U \cap V \), und \( U \cap V \) ist unter Skalarmultiplikation abgeschlossen.
    \end{enumerate}
    Daher ist \( U \cap V \) ein Untervektorraum von \( K^n \).

    \item \textbf{Untervektorraum \( U + V \):}
    \begin{enumerate}
        \item \textbf{Nullvektor:} Da \( U \) und \( V \) Untervektorräume sind, enthalten beide den Nullvektor \( 0 \). Setzen wir \( u = 0 \in U \) und \( v = 0 \in V \), dann ist \( u + v = 0 \), was zeigt, dass \( 0 \in U + V \).

        \item \textbf{Abgeschlossenheit unter Addition:} Sei \( x, y \in U + V \). Dann existieren \( u_1, u_2 \in U \) und \( v_1, v_2 \in V \) mit \( x = u_1 + v_1 \) und \( y = u_2 + v_2 \). Dann ist:
        \[
        x + y = (u_1 + v_1) + (u_2 + v_2) = (u_1 + u_2) + (v_1 + v_2).
        \]
        Da \( U \) und \( V \) unter Addition abgeschlossen sind, gilt \( u_1 + u_2 \in U \) und \( v_1 + v_2 \in V \). Somit ist \( x + y \in U + V \), und \( U + V \) ist unter Addition abgeschlossen.

        \item \textbf{Abgeschlossenheit unter Skalarmultiplikation:} Sei \( x \in U + V \) und \( c \in K \). Dann existieren \( u \in U \) und \( v \in V \) mit \( x = u + v \). Dann ist:
        \[
        c \cdot x = c \cdot (u + v) = (c \cdot u) + (c \cdot v).
        \]
        Da \( U \) und \( V \) unter Skalarmultiplikation abgeschlossen sind, gilt \( c \cdot u \in U \) und \( c \cdot v \in V \). Somit ist \( c \cdot x \in U + V \), und \( U + V \) ist unter Skalarmultiplikation abgeschlossen.
    \end{enumerate}
    Daher ist \( U + V \) ein Untervektorraum von \( K^n \).
\end{enumerate}

\subsection*{Teil (b)}

\textbf{Beweis:} Wir zeigen, dass die Menge \( U \cup V \) genau dann ein Untervektorraum von \( K^n \) ist, wenn \( U \subseteq V \) oder \( V \subseteq U \) gilt.

\begin{enumerate}
    \item \textbf{Notwendigkeit:} Angenommen, \( U \cup V \) ist ein Untervektorraum von \( K^n \). Wenn weder \( U \subseteq V \) noch \( V \subseteq U \) gilt, dann existieren Vektoren \( u \in U \setminus V \) und \( v \in V \setminus U \). Da \( U \cup V \) ein Untervektorraum ist, muss \( u + v \in U \cup V \) gelten. Da \( u \notin V \) und \( v \notin U \), kann \( u + v \) weder in \( U \) noch in \( V \) liegen, was im Widerspruch zur Definition von \( U \cup V \) als Untervektorraum steht. Daher muss entweder \( U \subseteq V \) oder \( V \subseteq U \) gelten.

    \item \textbf{Hinreichend:} Angenommen, \( U \subseteq V \). Dann gilt \( U \cup V = V \), und da \( V \) ein Untervektorraum ist, ist auch \( U \cup V \) ein Untervektorraum. Analog gilt, wenn \( V \subseteq U \), dann ist \( U \cup V = U \), und \( U \cup V \) ist ebenfalls ein Untervektorraum.
\end{enumerate}

Damit ist gezeigt, dass \( U \cup V \) genau dann ein Untervektorraum von \( K^n \) ist, wenn \( U \subseteq V \) oder \( V \subseteq U \) gilt.


\section*{Aufgabe 4.4}

Prüfen Sie, ob die folgenden Teilmengen linear unabhängig sind:

\subsection*{Teil (a)}

Gegeben sei der Körper \( K = \mathbb{Q} \) und die Menge
\[
M_1 := \{ 0 \} \subset \mathbb{Q}^4 =: V.
\]

\textbf{Beweis:} Die Menge \( M_1 \) besteht nur aus dem Nullvektor. Eine Menge, die nur den Nullvektor enthält, ist per Definition \textbf{nicht linear unabhängig}. Der Nullvektor ist immer linear abhängig, da ein Vielfaches des Nullvektors immer den Nullvektor ergibt.

\textbf{Ergebnis:} Die Menge \( M_1 \) ist \textbf{nicht linear unabhängig}.

\subsection*{Teil (b)}

Gegeben sei der Körper \( K = \mathbb{R} \) und die Menge
\[
M_2 := \left\{ \begin{pmatrix} t \\ 2t \end{pmatrix} : t \in \mathbb{R} \right\} \subset \mathbb{R}^2 =: V.
\]

\textbf{Beweis:} Die Menge \( M_2 \) besteht aus Vektoren der Form \( \begin{pmatrix} t \\ 2t \end{pmatrix} \) mit \( t \in \mathbb{R} \). Diese Vektoren sind alle Vielfache des Vektors \( \begin{pmatrix} 1 \\ 2 \end{pmatrix} \). Daher ist \( M_2 \) ein eindimensionaler Untervektorraum von \( \mathbb{R}^2 \), der durch den Vektor \( \begin{pmatrix} 1 \\ 2 \end{pmatrix} \) aufgespannt wird.

Da jeder Vektor in \( M_2 \) als Vielfaches von \( \begin{pmatrix} 1 \\ 2 \end{pmatrix} \) dargestellt werden kann, sind die Vektoren in \( M_2 \) \textbf{linear abhängig}.

\textbf{Ergebnis:} Die Menge \( M_2 \) ist \textbf{nicht linear unabhängig}.

\subsection*{Teil (c)}

Gegeben sei der Körper \( K = \mathbb{C} \) und die Menge
\[
M_3 := \left\{ \begin{pmatrix} i \\ 1 \\ -i \end{pmatrix}, \begin{pmatrix} 0 \\ 1 + i \\ 1 \end{pmatrix}, \begin{pmatrix} 1 \\ 0 \\ i - 1 \end{pmatrix} \right\} \subset \mathbb{C}^3 =: V.
\]

\textbf{Beweis:} Um zu prüfen, ob die Vektoren in \( M_3 \) linear unabhängig sind, untersuchen wir, ob es Skalare \( \lambda_1 \), \( \lambda_2 \) und \( \lambda_3 \in \mathbb{C} \) gibt, sodass:
\[
\lambda_1 \begin{pmatrix} i \\ 1 \\ -i \end{pmatrix} + \lambda_2 \begin{pmatrix} 0 \\ 1 + i \\ 1 \end{pmatrix} + \lambda_3 \begin{pmatrix} 1 \\ 0 \\ i - 1 \end{pmatrix} = \begin{pmatrix} 0 \\ 0 \\ 0 \end{pmatrix}.
\]

Diese Gleichung lässt sich in das folgende lineare Gleichungssystem umschreiben:
\[
\begin{cases}
\lambda_1 \cdot i + \lambda_3 = 0, \\
\lambda_1 + \lambda_2 \cdot (1 + i) = 0, \\
-\lambda_1 \cdot i + \lambda_2 + \lambda_3 \cdot (i - 1) = 0.
\end{cases}
\]

Wir lösen dieses Gleichungssystem mit dem Gausschen Eliminiationsverfahren und finden, dass die einzige Lösung \( \lambda_1 = \lambda_2 = \lambda_3 = 0 \) ist. Dies bedeutet, dass keine nicht-triviale Linearkombination der Vektoren den Nullvektor ergibt. Daher sind die Vektoren in \( M_3 \) linear unabhängig.

\textbf{Ergebnis:} Die Vektoren in \( M_3 \) sind \textbf{linear unabhängig}.

\subsection*{Teil (d)}

Gegeben sei der Körper \( K = \mathbb{F}_5 \) (der endliche Körper mit fünf Elementen) und die Menge
\[
M_4 := \left\{ \begin{pmatrix} 3 \\ 1 \\ 2 \end{pmatrix}, \begin{pmatrix} 1 \\ 4 \\ 2 \end{pmatrix} \right\} \subset \mathbb{F}_5^3 =: V.
\]

\textbf{Beweis:} Um zu prüfen, ob die Vektoren in \( M_4 \) linear unabhängig sind, stellen wir die Frage, ob es Skalare \( c_1, c_2 \in \mathbb{F}_5 \) gibt, sodass:
\[
c_1 \begin{pmatrix} 3 \\ 1 \\ 2 \end{pmatrix} + c_2 \begin{pmatrix} 1 \\ 4 \\ 2 \end{pmatrix} = \begin{pmatrix} 0 \\ 0 \\ 0 \end{pmatrix}.
\]

Dies ergibt das folgende Gleichungssystem in \( \mathbb{F}_5 \):
\[
\begin{cases}
3c_1 + c_2 \equiv 0 \pmod{5}, \\
c_1 + 4c_2 \equiv 0 \pmod{5}, \\
2c_1 + 2c_2 \equiv 0 \pmod{5}.
\end{cases}
\]

Wir lösen dieses System schrittweise im endlichen Körper \( \mathbb{F}_5 \):

\begin{enumerate}
    \item \textbf{Erste Gleichung:} \( 3c_1 + c_2 \equiv 0 \pmod{5} \Rightarrow c_2 \equiv -3c_1 \pmod{5} \). Da \( -3 \equiv 2 \pmod{5} \), erhalten wir \( c_2 \equiv 2c_1 \pmod{5} \).

    \item \textbf{Zweite Gleichung:} Setzen wir \( c_2 \equiv 2c_1 \pmod{5} \) in die zweite Gleichung ein:
    \[
    c_1 + 4 \cdot (2c_1) \equiv 0 \pmod{5} \Rightarrow c_1 + 8c_1 \equiv 0 \pmod{5} \Rightarrow 9c_1 \equiv 0 \pmod{5}.
    \]
    Da \( 9 \equiv 4 \pmod{5} \), haben wir \( 4c_1 \equiv 0 \pmod{5} \). Da \( 4 \) in \( \mathbb{F}_5 \) eine Einheit ist, folgt \( c_1 = 0 \).

    \item \textbf{Einsetzen in die erste Gleichung:} Setzen wir \( c_1 = 0 \) in die erste Gleichung ein, ergibt sich \( c_2 = 0 \).
\end{enumerate}

Da die einzige Lösung \( c_1 = 0 \) und \( c_2 = 0 \) ist, sind die Vektoren in \( M_4 \) linear unabhängig.

\textbf{Ergebnis:} Die Vektoren in \( M_4 \) sind \textbf{linear unabhängig}.

\bibliography{main}
\bibliographystyle{plain}

\end{document}
