%! Author = felix
%! Date = 11.10.2024

% Preamble
\documentclass[11pt]{article}

% Packages
\usepackage{amsmath}
\usepackage{amsfonts}

% Document
\begin{document}

\title{Übungsblatt 4}
\author{Felix Kleine Bösing}
\maketitle

\section*{Aufgabe 1}

Untersuchen Sie, welche der folgenden Teilmengen Untervektorräume von \( \mathbb{Q}^3 \) sind:
\[
\begin{aligned}
    M_1 &= \{(x, y, z) \in \mathbb{Q}^3 : x, y, z \geq 0\}, \\
    M_2 &= \{(x, y, z) \in \mathbb{Q}^3 : 3x + y + z = 5\}, \\
    M_3 &= \{(x, y, z) \in \mathbb{Q}^3 : x + 2y = 3z\}, \\
    M_4 &= \{(x, y, z) \in \mathbb{Q}^3 : xy - z = 0\}.
\end{aligned}
\]

\subsection*{Teil (a)}

\textbf{Beweis:} Um zu überprüfen, ob \( M_1 \) ein Untervektorraum von \( \mathbb{Q}^3 \) ist, müssen wir die folgenden Eigenschaften zeigen:

\begin{enumerate}
    \item \textbf{Der Nullvektor muss enthalten sein:} Der Nullvektor in \( \mathbb{Q}^3 \) ist \((0, 0, 0)\). Da \(0 \geq 0\) für jede Komponente gilt, gehört der Nullvektor zu \( M_1 \).

    \item \textbf{Abgeschlossenheit unter Addition:} Nehmen wir an, dass \( (x_1, y_1, z_1), (x_2, y_2, z_2) \in M_1 \). Dann sind \( x_1, y_1, z_1 \geq 0 \) und \( x_2, y_2, z_2 \geq 0 \). Für die Summe \((x_1 + x_2, y_1 + y_2, z_1 + z_2)\) gilt ebenfalls \( x_1 + x_2 \geq 0 \), \( y_1 + y_2 \geq 0 \) und \( z_1 + z_2 \geq 0 \), sodass die Summe auch in \( M_1 \) liegt.

    \item \textbf{Abgeschlossenheit unter Skalarmultiplikation:} Sei \( (x, y, z) \in M_1 \) und \( c \in \mathbb{Q} \). Wenn \( c < 0 \), dann wird eine oder mehrere der Komponenten \( cx, cy, cz \) negativ, was die Bedingung \( x, y, z \geq 0 \) verletzt. Daher ist \( M_1 \) nicht unter Skalarmultiplikation abgeschlossen.
\end{enumerate}

Da \( M_1 \) nicht unter Skalarmultiplikation abgeschlossen ist, ist es \textbf{kein Untervektorraum} von \( \mathbb{Q}^3 \).

\subsection*{Teil (b)}

\textbf{Beweis:} Untersuchen wir, ob \( M_2 \) ein Untervektorraum von \( \mathbb{Q}^3 \) ist.

\begin{enumerate}
    \item \textbf{Der Nullvektor muss enthalten sein:} Der Nullvektor in \( \mathbb{Q}^3 \) ist \((0, 0, 0)\). Setzen wir diesen in die Bedingung \(3x + y + z = 5\) ein, so erhalten wir:
   \[
   3 \cdot 0 + 0 + 0 = 0 \neq 5.
   \]
   Daher gehört der Nullvektor \textbf{nicht} zu \( M_2 \).
\end{enumerate}

Da der Nullvektor nicht in \( M_2 \) liegt, ist \( M_2 \) \textbf{kein Untervektorraum} von \( \mathbb{Q}^3 \).

\subsection*{Teil (c)}

\textbf{Beweis:} Untersuchen wir, ob \( M_3 \) ein Untervektorraum von \( \mathbb{Q}^3 \) ist.

\begin{enumerate}
    \item \textbf{Der Nullvektor muss enthalten sein:} Der Nullvektor in \( \mathbb{Q}^3 \) ist \((0, 0, 0)\). Setzen wir diesen in die Bedingung \(x + 2y = 3z\) ein, so erhalten wir:
   \[
   0 + 2 \cdot 0 = 3 \cdot 0,
   \]
   was offensichtlich wahr ist. Daher gehört der Nullvektor zu \( M_3 \).

    \item \textbf{Abgeschlossenheit unter Addition:} Nehmen wir an, dass \( (x_1, y_1, z_1), (x_2, y_2, z_2) \in M_3 \). Dann gilt:
   \[
   x_1 + 2y_1 = 3z_1 \quad \text{und} \quad x_2 + 2y_2 = 3z_2.
   \]
   Für die Summe \((x_1 + x_2, y_1 + y_2, z_1 + z_2)\) ergibt sich:
   \[
   (x_1 + x_2) + 2(y_1 + y_2) = (x_1 + 2y_1) + (x_2 + 2y_2) = 3z_1 + 3z_2 = 3(z_1 + z_2),
   \]
   sodass die Summe ebenfalls die Bedingung erfüllt. \( M_3 \) ist also unter Addition abgeschlossen.

    \item \textbf{Abgeschlossenheit unter Skalarmultiplikation:} Sei \( (x, y, z) \in M_3 \) und \( c \in \mathbb{Q} \). Dann gilt:
   \[
   x + 2y = 3z.
   \]
   Für das Produkt \( c \cdot (x, y, z) = (cx, cy, cz) \) erhalten wir:
   \[
   cx + 2(cy) = c(x + 2y) = c \cdot 3z = 3(cz),
   \]
   was zeigt, dass auch \( (cx, cy, cz) \) die Bedingung erfüllt. Somit ist \( M_3 \) unter Skalarmultiplikation abgeschlossen.
\end{enumerate}

Da \( M_3 \) sowohl den Nullvektor enthält als auch unter Addition und Skalarmultiplikation abgeschlossen ist, ist \( M_3 \) ein \textbf{Untervektorraum} von \( \mathbb{Q}^3 \).

\subsection*{Teil (d)}

\textbf{Beweis:} Untersuchen wir, ob \( M_4 \) ein Untervektorraum von \( \mathbb{Q}^3 \) ist.

\begin{enumerate}
    \item \textbf{Der Nullvektor muss enthalten sein:} Der Nullvektor in \( \mathbb{Q}^3 \) ist \((0, 0, 0)\). Setzen wir diesen in die Bedingung \(xy - z = 0\) ein, so erhalten wir:
   \[
   0 \cdot 0 - 0 = 0,
   \]
   was offensichtlich wahr ist. Daher gehört der Nullvektor zu \( M_4 \).

    \item \textbf{Abgeschlossenheit unter Addition:} Nehmen wir an, dass \( (x_1, y_1, z_1), (x_2, y_2, z_2) \in M_4 \), also \( x_1 y_1 = z_1 \) und \( x_2 y_2 = z_2 \). Für die Summe \((x_1 + x_2, y_1 + y_2, z_1 + z_2)\) ergibt sich jedoch:
   \[
   (x_1 + x_2)(y_1 + y_2) = x_1 y_1 + x_1 y_2 + x_2 y_1 + x_2 y_2.
   \]
   Da zusätzliche Kreuzterme wie \( x_1 y_2 \) und \( x_2 y_1 \) auftreten, ist im Allgemeinen \((x_1 + x_2)(y_1 + y_2) \neq z_1 + z_2\). Somit ist \( M_4 \) nicht unter Addition abgeschlossen.
\end{enumerate}

Da \( M_4 \) nicht unter Addition abgeschlossen ist, ist es \textbf{kein Untervektorraum} von \( \mathbb{Q}^3 \).

\section*{Aufgabe 4.2}

\subsection*{Teil (a)}

\textbf{Beweis:} Wir sollen ein Beispiel eines Vektorraums \( V \) und einer Teilmenge \( M \subseteq V \) finden, sodass für alle \( v, w \in M \) mit \( v \neq w \) die Vektoren \( v \) und \( w \) linear unabhängig sind, die Menge \( M \) jedoch linear abhängig ist.

Ein solches Beispiel ist der Vektorraum \( V = \mathbb{R}^3 \) und die Teilmenge
\[
M = \left\{ \begin{pmatrix} 1 \\ 0 \\ 0 \end{pmatrix}, \begin{pmatrix} 0 \\ 1 \\ 0 \end{pmatrix}, \begin{pmatrix} 1 \\ 1 \\ 0 \end{pmatrix} \right\}.
\]

Betrachten wir die Eigenschaften dieser Vektoren:

\begin{enumerate}
    \item Für jedes Paar unterschiedlicher Vektoren \( v, w \in M \) sind \( v \) und \( w \) linear unabhängig. Zum Beispiel sind \( \begin{pmatrix} 1 \\ 0 \\ 0 \end{pmatrix} \) und \( \begin{pmatrix} 0 \\ 1 \\ 0 \end{pmatrix} \) linear unabhängig, da keine Linearkombination \( c_1 \begin{pmatrix} 1 \\ 0 \\ 0 \end{pmatrix} + c_2 \begin{pmatrix} 0 \\ 1 \\ 0 \end{pmatrix} = 0 \) für \( c_1, c_2 \in \mathbb{R} \) außer \( c_1 = c_2 = 0 \) existiert.

    \item Die Menge \( M \) ist jedoch linear abhängig, da
    \[
    \begin{pmatrix} 1 \\ 0 \\ 0 \end{pmatrix} + \begin{pmatrix} 0 \\ 1 \\ 0 \end{pmatrix} - \begin{pmatrix} 1 \\ 1 \\ 0 \end{pmatrix} = \begin{pmatrix} 0 \\ 0 \\ 0 \end{pmatrix}.
    \]
    Dies zeigt, dass die Vektoren in \( M \) eine lineare Abhängigkeit aufweisen.
\end{enumerate}

Somit erfüllt die Teilmenge \( M \) die Bedingungen der Aufgabe.

\subsection*{Teil (b)}

\textbf{Beweis:} Gegeben sei ein Körper \( K \) und ein \( K \)-Vektorraum \( V \). Sei \( M = \{v_1, \dots, v_n\} \subseteq V \) eine Teilmenge, wobei \( 0 \notin M \). Wir sollen zeigen, dass \( M \) genau dann linear unabhängig ist, wenn für alle \( i \in \{1, 2, \dots, n-1\} \) gilt:
\[
\langle v_1, \dots, v_i \rangle \cap \langle v_{i+1}, \dots, v_n \rangle = \{0\}.
\]

\begin{enumerate}
    \item \textbf{Notwendigkeit:} Angenommen, \( M \) ist linear unabhängig. Dann bedeutet dies, dass keine nicht-triviale Linearkombination der Vektoren in \( M \) den Nullvektor ergibt. Insbesondere ist jeder Vektor \( v_i \) nicht in der Linearkombination der anderen Vektoren, was impliziert, dass für jedes \( i \) die Schnittmenge \( \langle v_1, \dots, v_i \rangle \cap \langle v_{i+1}, \dots, v_n \rangle \) nur den Nullvektor enthält, also
    \[
    \langle v_1, \dots, v_i \rangle \cap \langle v_{i+1}, \dots, v_n \rangle = \{0\}.
    \]

    \item \textbf{Hinreichend:} Angenommen, für alle \( i \in \{1, 2, \dots, n-1\} \) gilt \( \langle v_1, \dots, v_i \rangle \cap \langle v_{i+1}, \dots, v_n \rangle = \{0\} \). Dies bedeutet, dass es keine nicht-triviale Linearkombination von \( v_1, \dots, v_i \) gibt, die auch als Linearkombination von \( v_{i+1}, \dots, v_n \) ausgedrückt werden kann. Folglich ist \( M \) linear unabhängig, da jede Linearkombination, die den Nullvektor ergibt, nur die triviale Lösung \( c_1 = c_2 = \dots = c_n = 0 \) hat.
\end{enumerate}

Damit ist gezeigt, dass \( M \) genau dann linear unabhängig ist, wenn für alle \( i \in \{1, 2, \dots, n-1\} \) gilt:
\[
\langle v_1, \dots, v_i \rangle \cap \langle v_{i+1}, \dots, v_n \rangle = \{0\}.
\]




\bibliography{main}
\bibliographystyle{plain}

\end{document}
