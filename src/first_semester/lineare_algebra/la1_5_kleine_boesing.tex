%! Author = felix
%! Date = 11.10.2024

% Preamble
\documentclass[11pt]{article}

% Packages
\usepackage{amsmath}
\usepackage{amsfonts}

% Document
\begin{document}

\title{Übungsblatt 5}
\author{Felix Kleine Bösing}
\maketitle

\section*{Aufgabe 1}

\
\subsection*{Teil (a)}

\textbf{Beweis:} Wir beweisen die Aussage per vollständiger Induktion. Sei
\[
P(n) : \sum_{k=1}^n k^2 = \frac{n(n+1)(2n+1)}{6}
\]

\begin{enumerate}
    \item \textbf{Induktionsanfang:} Für \( n = 2 \) gilt:
    \[
    \sum_{k=1}^2 k^2 = 1^2 + 2^2 = 1 + 4 = 5
    \]
    und
    \[
    \frac{2 \cdot (2+1) \cdot (2 \cdot 2 + 1)}{6} = \frac{2 \cdot 3 \cdot 5}{6} = 5.
    \]
    Somit stimmt die Gleichung für \( n = 2 \).

    \item \textbf{Induktionsvoraussetzung:} Wir nehmen an, dass die Aussage für ein beliebiges \( n \geq 2 \) gilt, also:
    \[
    \sum_{k=1}^n k^2 = \frac{n(n+1)(2n+1)}{6}.
    \]

    \item \textbf{Induktionsschritt:} Es ist zu zeigen, dass die Aussage dann auch für \( n+1 \) gilt, also:
    \[
    \sum_{k=1}^{n+1} k^2 = \frac{(n+1)(n+2)(2(n+1)+1)}{6} = \frac{(n+1)(n+2)(2n+3)}{6}.
    \]
    Wir schreiben:
    \[
    \sum_{k=1}^{n+1} k^2 = \sum_{k=1}^n k^2 + (n+1)^2.
    \]
    Nach Induktionsvoraussetzung ergibt sich:
    \[
    \sum_{k=1}^{n+1} k^2 = \frac{n(n+1)(2n+1)}{6} + (n+1)^2.
    \]
    \[
    \sum_{k=1}^{n+1} k^2 = \frac{(n+1)(2n^2+n)}{6} + \frac{6(n+1)^2}{6}.
    \]
    \[
    \sum_{k=1}^{n+1} k^2 = \frac{(n+1)((2n^2+n)+6(n+1))}{6}.
    \]
    \[
    \sum_{k=1}^{n+1} k^2 = \frac{(n+1)(2n^2+7n+6)}{6}.
    \]

    Nach Umforung ergibt sich die folgende Aussage.
    \[
    \sum_{k=1}^{n+1} k^2 = \frac{(n+1)(n+2)(2n+3)}{6}.
    \]
    Somit ist die Aussage für \( n+1 \) wahr, und der Induktionsschritt ist abgeschlossen.
\end{enumerate}
Damit ist die Aussage per vollständiger Induktion bewiesen.

\subsection*{Teil (b)}

\textbf{Beweis:} Wir beweisen die Aussage ebenfalls per vollständiger Induktion. Sei
\[
Q(n) : \prod_{k=2}^n \left(1 - \frac{1}{k}\right) = \frac{1}{n}.
\]

\begin{enumerate}
    \item \textbf{Induktionsanfang:} Für \( n = 2 \) gilt:
    \[
    \prod_{k=2}^2 \left(1 - \frac{1}{k}\right) = 1 - \frac{1}{2} = \frac{1}{2}.
    \]
    Somit ist die Gleichung für \( n = 2 \) erfüllt.

    \item \textbf{Induktionsvoraussetzung:} Wir nehmen an, dass die Aussage für ein beliebiges \( n \geq 2 \) gilt, also:
    \[
    \prod_{k=2}^n \left(1 - \frac{1}{k}\right) = \frac{1}{n}.
    \]

    \item \textbf{Induktionsschritt:} Es ist zu zeigen, dass die Aussage dann auch für \( n+1 \) gilt, also:
    \[
    \prod_{k=2}^{n+1} \left(1 - \frac{1}{k}\right) = \frac{1}{n+1}.
    \]
    Wir schreiben:
    \[
    \prod_{k=2}^{n+1} \left(1 - \frac{1}{k}\right) = \left(\prod_{k=2}^n \left(1 - \frac{1}{k}\right)\right) \cdot \left(1 - \frac{1}{n+1}\right).
    \]
    Nach Induktionsvoraussetzung ergibt sich:
    \[
    \prod_{k=2}^{n+1} \left(1 - \frac{1}{k}\right) = \frac{1}{n} \cdot \left(1 - \frac{1}{n+1}\right).
    \]
    Diesen Ausdruck formen wir um und erhalten:
    \[
    \frac{1}{n} \cdot \frac{n}{n+1} = \frac{1}{n+1}.
    \]
    Somit ist die Aussage für \( n+1 \) wahr, und der Induktionsschritt ist abgeschlossen.
\end{enumerate}
Damit ist die Aussage per vollständiger Induktion bewiesen.


\section*{Aufgabe 2}

Es sei \( K \) ein Körper mit endlich vielen Elementen und \( q := |K| \) die Anzahl der Elemente von \( K \).

\subsection*{Teil (a)}

\textbf{Beweis:}

Wir zeigen, dass ein endlich erzeugter \( K \)-Vektorraum \( V \) der Dimension \( n \) genau \( q^n \) Elemente besitzt.

\begin{enumerate}
    \item Da \( V \) ein Vektorraum der Dimension \( n \) über dem Körper \( K \) ist, existiert eine Basis von \( V \) mit genau \( n \) Vektoren.
    \item Jedes Element von \( V \) kann eindeutig als Linearkombination der Basisvektoren dargestellt werden:
    \[
    v = a_1 b_1 + a_2 b_2 + \dots + a_n b_n,
    \]
    wobei \( a_i \in K \) die Koeffizienten und \( b_i \) die Basisvektoren sind.
    \item Da \( K \) genau \( q \) Elemente besitzt, hat jede der \( n \) Koeffizienten \( a_i \) genau \( q \) mögliche Werte.
    \item Somit gibt es insgesamt \( q \times q \times \dots \times q = q^n \) verschiedene Kombinationen der Koeffizienten, was \( q^n \) verschiedene Elemente in \( V \) ergibt.
\end{enumerate}
Damit ist gezeigt, dass \( V \) genau \( q^n \) Elemente besitzt.

\subsection*{Teil (b)}

\textbf{Beweis:}

Wir zeigen, dass der \( K \)-Vektorraum \( K^2 \) genau \( (q^2 - 1)(q^2 - q) \) geordnete Basen besitzt.

\begin{enumerate}
    \item Der Raum \( K^2 \) hat die Dimension 2, also besteht jede Basis von \( K^2 \) aus genau zwei Vektoren.
    \item Ein geordneter Vektor \( (v_1, v_2) \in K^2 \setminus \{(0, 0)\} \) kann als erster Basisvektor gewählt werden. Da dieser Vektor nicht der Nullvektor sein darf, gibt es \( q^2 - 1 \) Möglichkeiten für \( v_1 \).
    \item Für den zweiten Basisvektor \( v_2 \) muss gelten, dass \( v_2 \) nicht in der Richtung von \( v_1 \) liegt, um die Linearunabhängigkeit zu gewährleisten.
    \item Es gibt insgesamt \( q \) skalare Vielfache von \( v_1 \) (einschließlich des Nullvektors), die für \( v_2 \) nicht gewählt werden können.
    \item Daher gibt es \( q^2 - q \) mögliche Werte für \( v_2 \), die nicht in der Richtung von \( v_1 \) liegen.
\end{enumerate}
Damit gibt es insgesamt \( (q^2 - 1)(q^2 - q) \) geordnete Basen in \( K^2 \).

\section*{3}

Es sei \( K \) ein Körper mit Charakteristik verschieden von 2, \( V \) ein \( K \)-Vektorraum und \( (v_1, v_2, v_3, v_4, v_5) \) eine Basis von \( V \).

Entscheiden Sie, welche der folgenden Systeme linear unabhängig sind und welche den gesamten Vektorraum \( V \) erzeugen.

\subsection*{Teil (a)}

System: \( (v_1 + v_2, v_1 + v_3, v_1 + v_4, v_1 + v_5) \)

\textbf{Analyse:}

\begin{enumerate}
    \item Da \( (v_1, v_2, v_3, v_4, v_5) \) eine Basis von \( V \) ist, sind diese Vektoren linear unabhängig und erzeugen den gesamten Raum \( V \).
    \item Da wir im System nur vier Vektoren haben, kann dieses System nicht den gesamten Vektorraum \( V \) erzeugen, da \( V \) eine Dimension von 5 hat.
    \item Zur Prüfung der linearen Unabhängigkeit prüfen wir, ob eine Linearkombination \( a_1 (v_1 + v_2) + a_2 (v_1 + v_3) + a_3 (v_1 + v_4) + a_4 (v_1 + v_5) = 0 \) nur die triviale Lösung \( a_1 = a_2 = a_3 = a_4 = 0 \) hat. Da dies nicht immer der Fall ist, ist das System linear abhängig.
\end{enumerate}
\textbf{Ergebnis:} Das System ist linear abhängig und erzeugt nicht den gesamten Raum \( V \).

\subsection*{Teil (b)}

System: \( (v_1, v_2, v_3 + v_4 + v_5) \)

\textbf{Analyse:}

\begin{enumerate}
    \item Das System enthält nur drei Vektoren. Da \( V \) eine Dimension von 5 hat, kann dieses System nicht den gesamten Vektorraum \( V \) erzeugen.
    \item Zur Prüfung der linearen Unabhängigkeit prüfen wir, ob eine Linearkombination \( b_1 v_1 + b_2 v_2 + b_3 (v_3 + v_4 + v_5) = 0 \) nur die triviale Lösung \( b_1 = b_2 = b_3 = 0 \) hat. Da die Vektoren in einer Basis linear unabhängig sind, ist auch dieser Vektorraum unabhängig.
\end{enumerate}
\textbf{Ergebnis:} Das System ist linear unabhängig, erzeugt aber nicht den gesamten Raum \( V \).

\subsection*{Teil (c)}

System: \( (v_1, v_2, v_1 + v_2, v_3, v_4) \)

\textbf{Analyse:}

\begin{enumerate}
    \item Da dieses System fünf Vektoren enthält, könnte es theoretisch den gesamten Raum \( V \) erzeugen.
    \item Zur Prüfung der linearen Unabhängigkeit stellen wir fest, dass \( v_1 + v_2 \) als Linearkombination von \( v_1 \) und \( v_2 \) dargestellt werden kann. Somit ist das System linear abhängig.
\end{enumerate}
\textbf{Ergebnis:} Das System ist linear abhängig und erzeugt daher nicht den gesamten Raum \( V \).

\subsection*{Teil (d)}

System: \( (v_1 + v_2, v_2 + v_3, v_3 + v_4, v_4 + v_5, v_5 + v_1) \)

\textbf{Analyse:}

\begin{enumerate}
    \item Das System enthält fünf Vektoren, was der Dimension des Vektorraums \( V \) entspricht, sodass es potenziell den gesamten Raum \( V \) erzeugen könnte.
    \item Wir prüfen, ob die Vektoren linear unabhängig sind, indem wir annehmen, dass \( c_1 (v_1 + v_2) + c_2 (v_2 + v_3) + c_3 (v_3 + v_4) + c_4 (v_4 + v_5) + c_5 (v_5 + v_1) = 0 \) nur die triviale Lösung \( c_1 = c_2 = c_3 = c_4 = c_5 = 0 \) hat.
    \item Da jeder Vektor eine Linearkombination der Basisvektoren ist und alle Basisvektoren involviert sind, kann man zeigen, dass dieses System linear unabhängig ist und den gesamten Raum \( V \) erzeugt.
\end{enumerate}
\textbf{Ergebnis:} Das System ist linear unabhängig und erzeugt den gesamten Raum \( V \).

\section*{Aufgabe 4}

Sei \( K \) ein Körper, \( a = (a_1, \dots, a_n) \), \( b = (b_1, \dots, b_n) \in K^n \). Zeigen Sie, dass \( a \) und \( b \) genau dann linear unabhängig sind, wenn \( a_i b_j - a_j b_i \neq 0 \) für mindestens ein Paar \( (i, j) \).

\textbf{Beweis:}

\begin{enumerate}
    \item Zwei Vektoren \( a \) und \( b \) in einem \( K \)-Vektorraum \( K^n \) sind linear abhängig, wenn es Skalare \( \lambda, \mu \in K \), nicht beide null, gibt, so dass:
    \[
    \lambda a + \mu b = 0.
    \]
    Das bedeutet, dass für jedes \( k = 1, \dots, n \) gilt:
    \[
    \lambda a_k + \mu b_k = 0.
    \]

    \item Angenommen, \( a \) und \( b \) seien linear abhängig. Dann gibt es ein \( \lambda \in K \setminus \{0\} \), so dass \( a = \lambda b \) oder \( b = \lambda a \). Das bedeutet, dass die Komponenten \( a_i \) und \( b_i \) für jedes \( i \) im Verhältnis \( a_i = \lambda b_i \) stehen.

    \item Wenn \( a \) und \( b \) linear unabhängig sind, dann existiert kein solches \( \lambda \in K \), und daher muss es mindestens ein Paar \( (i, j) \) geben, so dass \( \frac{a_i}{b_i} \neq \frac{a_j}{b_j} \) (wobei Division im Körper \( K \) existiert, weil \( b_i \) und \( b_j \) ungleich null sind).

    \item Dies ist äquivalent dazu, dass \( a_i b_j - a_j b_i \neq 0 \) für mindestens ein Paar \( (i, j) \), denn wenn \( \frac{a_i}{b_i} = \frac{a_j}{b_j} \), dann wäre \( a_i b_j = a_j b_i \).
\end{enumerate}

\textbf{Ergebnis:} Die Vektoren \( a \) und \( b \) sind genau dann linear unabhängig, wenn \( a_i b_j - a_j b_i \neq 0 \) für mindestens ein Paar \( (i, j) \).

\bibliography{main}
\bibliographystyle{plain}

\end{document}
