%! Author = felix
%! Date = 11.10.2024

% Preamble
\documentclass[11pt]{article}

% Packages
\usepackage{amsmath}
\usepackage{amsfonts}

% Document
\begin{document}

\title{Übungsblatt 6}
\author{Felix Kleine Bösing}
\maketitle

\section*{Aufgabe 1}

Gegeben sei die Matrix \( A \in \mathbb{Q}^{5 \times 5} \) mit
\[
A =
\begin{pmatrix}
1 & 0 & 1 & 0 & 1 \\
0 & 1 & 1 & 1 & 0 \\
1 & 1 & 1 & 1 & 1 \\
0 & 1 & 1 & 1 & 0 \\
1 & 0 & 1 & 0 & 1
\end{pmatrix}
\]

Berechnen Sie:
\begin{enumerate}
    \item den Rang der Matrix,
    \item eine Basis des von den Zeilen der Matrix aufgespannten Untervektorraums von \( \mathbb{Q}^5 \),
    \item sowie die Dimension des Kerns \( \ker(A) \), also \( \dim(L_{A, 0}) \)
\end{enumerate}

\subsection*{Lösung}

\textbf{Beweis:} Wir gehen schrittweise vor.

\begin{enumerate}
    \item \textbf{Berechnung des Rangs der Matrix \( A \):} \\
    Um den Rang der Matrix \( A \) zu bestimmen, bringen wir \( A \) mittels elementarer Zeilenumformungen in reduzierte Zeilenstufenform.

    \textbf{Schritt 1: Abhängige Zeilen entfernen.} \\
    Die Zeilen der Matrix \( A \) sind:
    \[
    \begin{aligned}
        &\mathbf{z}_1 = (1, 0, 1, 0, 1), \quad \mathbf{z}_2 = (0, 1, 1, 1, 0), \\
        &\mathbf{z}_3 = (1, 1, 1, 1, 1), \quad \mathbf{z}_4 = (0, 1, 1, 1, 0), \quad \mathbf{z}_5 = (1, 0, 1, 0, 1)
    \end{aligned}
    \]
    Es ist direkt erkennbar:
    \begin{itemize}
        \item \( \mathbf{z}_4 = \mathbf{z}_2 \) (Zeile 4 ist linear abhängig von Zeile 2),
        \item \( \mathbf{z}_5 = \mathbf{z}_1 \) (Zeile 5 ist linear abhängig von Zeile 1)
    \end{itemize}
    Wir entfernen \( \mathbf{z}_4 \) und \( \mathbf{z}_5 \), wodurch die reduzierte Matrix lautet:
    \[
    A' =
    \begin{pmatrix}
    1 & 0 & 1 & 0 & 1 \\
    0 & 1 & 1 & 1 & 0 \\
    1 & 1 & 1 & 1 & 1
    \end{pmatrix}
    \]

    \textbf{Schritt 2: Gauß-Elimination.} \\
    Wir bringen \( A' \) durch Zeilenoperationen in Zeilenstufenform:
    \[
    A' =
    \begin{pmatrix}
    1 & 0 & 1 & 0 & 1 \\
    0 & 1 & 1 & 1 & 0 \\
    1 & 1 & 1 & 1 & 1
    \end{pmatrix}
    \]
    Subtrahiere \( \mathbf{z}_1 \) von \( \mathbf{z}_3 \):
    \[
    \begin{pmatrix}
    1 & 0 & 1 & 0 & 1 \\
    0 & 1 & 1 & 1 & 0 \\
    0 & 1 & 0 & 1 & 0
    \end{pmatrix}
    \]
    Subtrahiere \( \mathbf{z}_2 \) von \( \mathbf{z}_3 \):
    \[
    \begin{pmatrix}
    1 & 0 & 1 & 0 & 1 \\
    0 & 1 & 1 & 1 & 0 \\
    0 & 0 & -1 & 0 & 0
    \end{pmatrix}
    \]
    Multipliziere die letzte Zeile mit \( -1 \):
    \[
    \begin{pmatrix}
    1 & 0 & 1 & 0 & 1 \\
    0 & 1 & 1 & 1 & 0 \\
    0 & 0 & 1 & 0 & 0
    \end{pmatrix}
    \]

    In Zeilenstufenform hat die Matrix \( 3 \) nicht-null Zeilen. Somit ist der Rang von \( A \):
    \[
    \text{Rang}(A) = 3
    \]

    \item \textbf{Bestimmung einer Basis des von den Zeilen aufgespannten Untervektorraums:} \\
    Die linear unabhängigen Zeilenvektoren der Zeilenstufenform von \( A' \) bilden eine Basis des Zeilenraums:
    \[
    \mathcal{B}_U = \left\{
    \begin{pmatrix} 1 & 0 & 1 & 0 & 1 \end{pmatrix}, \,
    \begin{pmatrix} 0 & 1 & 1 & 1 & 0 \end{pmatrix}, \,
    \begin{pmatrix} 0 & 0 & 1 & 0 & 0 \end{pmatrix}
    \right\}
    \]

    \item \textbf{Bestimmung der Dimension von \( \mathbf{L}_{A,0} := \ker(A) \):} \\
    Um die Dimension des Kerns von \( A \) zu bestimmen, wenden wir den Rangsatz an. Dieser besagt:
    \[
    \dim(\ker(A)) + \text{Rang}(A) = n,
    \]
    wobei \( n \) die Anzahl der Spalten der Matrix ist. In diesem Fall ist \( n = 5 \). Da wir bereits bestimmt haben, dass der Rang der Matrix \( \text{Rang}(A) = 3 \) ist, ergibt sich:
    \[
    \dim(\ker(A)) = n - \text{Rang}(A) = 5 - 3 = 2.
    \]

    \textbf{Interpretation:} Die Dimension von \( \ker(A) \) gibt die Anzahl der linear unabhängigen Lösungen des homogenen Gleichungssystems \( A \mathbf{x} = 0 \) an. In diesem Fall ist \( \dim(\ker(A)) = 2 \), was bedeutet, dass der Kern von \( A \) ein zweidimensionaler Untervektorraum von \( \mathbb{Q}^5 \) ist.

\end{enumerate}

\textbf{Ergebnis:}
\begin{enumerate}
    \item Der Rang der Matrix ist \( 3 \).
    \item Eine Basis des Zeilenraums ist:
    \[
    \mathcal{B}_U = \left\{
    \begin{pmatrix} 1 & 0 & 1 & 0 & 1 \end{pmatrix}, \,
    \begin{pmatrix} 0 & 1 & 1 & 1 & 0 \end{pmatrix}, \,
    \begin{pmatrix} 0 & 0 & 1 & 0 & 0 \end{pmatrix}
    \right\}
    \]
    \item Die Dimension des Kerns ist \( \dim(\ker(A)) = 2 \)
\end{enumerate}


\section*{Aufgabe 2}

Es sei \( K \) ein Körper, \( U, V \) und \( W \) \( K \)-Vektorräume und \( f: V \to W \), \( g: W \to U \) lineare Abbildungen. Zeigen Sie, dass dann auch
\[
h := g \circ f : V \to U
\]
eine lineare Abbildung ist.

\subsection*{Lösung}

\textbf{Beweis:} Wir zeigen, dass die Verkettung \( h = g \circ f \) die Eigenschaften einer linearen Abbildung erfüllt. Dafür müssen wir zeigen:
\begin{enumerate}
    \item \( h(v_1 + v_2) = h(v_1) + h(v_2) \) für alle \( v_1, v_2 \in V \),
    \item \( h(\lambda v) = \lambda h(v) \) für alle \( v \in V \) und \( \lambda \in K \)
\end{enumerate}

\textbf{1. Additivität:} \\
Für \( v_1, v_2 \in V \) gilt:
\[
h(v_1 + v_2) = (g \circ f)(v_1 + v_2) = g(f(v_1 + v_2))
\]
Da \( f \) eine lineare Abbildung ist, gilt:
\[
f(v_1 + v_2) = f(v_1) + f(v_2)
\]
Einsetzen in die obige Gleichung ergibt:
\[
h(v_1 + v_2) = g(f(v_1) + f(v_2))
\]
Da \( g \) ebenfalls linear ist, gilt:
\[
g(f(v_1) + f(v_2)) = g(f(v_1)) + g(f(v_2))
\]
Somit erhalten wir:
\[
h(v_1 + v_2) = g(f(v_1)) + g(f(v_2)) = h(v_1) + h(v_2)
\]

\textbf{2. Homogenität:} \\
Für \( v \in V \) und \( \lambda \in K \) gilt:
\[
h(\lambda v) = (g \circ f)(\lambda v) = g(f(\lambda v))
\]
Da \( f \) linear ist, gilt:
\[
f(\lambda v) = \lambda f(v)
\]
Einsetzen ergibt:
\[
h(\lambda v) = g(\lambda f(v))
\]
Da \( g \) linear ist, gilt:
\[
g(\lambda f(v)) = \lambda g(f(v))
\]
Somit erhalten wir:
\[
h(\lambda v) = \lambda g(f(v)) = \lambda h(v)
\]

\textbf{Schlussfolgerung:} \\
Da \( h \) sowohl additiv als auch homogen ist, folgt, dass \( h \) eine lineare Abbildung ist.

\textbf{Ergebnis:} Die Verkettung \( h = g \circ f : V \to U \) ist linear.


\section*{Aufgabe 3}

\subsection*{(a)}

Es seien die folgenden Vektoren in \( \mathbb{Q}^4 \) gegeben:
\[
a = \begin{pmatrix} 1 \\ 2 \\ 0 \\ -3 \end{pmatrix}, \,
b = \begin{pmatrix} 0 \\ -1 \\ 1 \\ 0 \end{pmatrix}, \,
c = \begin{pmatrix} 2 \\ -4 \\ 3 \\ -1 \end{pmatrix}, \,
d = \begin{pmatrix} 1 \\ 0 \\ 1 \\ -2 \end{pmatrix}.
\]
Sei \( U \) der von \( a, b, c, d \) erzeugte Unterraum. Geben Sie eine Basis von \( U \) an.

\subsection*{Lösung}

Um eine Basis des Unterraums \( U \) zu bestimmen, überprüfen wir die lineare Unabhängigkeit der Vektoren \( a, b, c, d \). Dazu bilden wir die Matrix, deren Zeilen die Vektoren \( a, b, c, d \) sind, und bringen sie durch Gauß-Eliminationsverfahren in Zeilenstufenform. Die nicht-null Zeilen der Zeilenstufenform bilden dann eine Basis von \( U \).

\textbf{Matrix der Vektoren:}
\[
A = \begin{pmatrix}
1 & 2 & 0 & -3 \\
0 & -1 & 1 & 0 \\
2 & -4 & 3 & -1 \\
1 & 0 & 1 & -2
\end{pmatrix}.
\]

\textbf{Schritt 1: Gauß-Eliminationsverfahren.}

\begin{enumerate}
    \item Zunächst verwenden wir die erste Zeile als Pivotzeile und eliminieren den Eintrag in der ersten Spalte der dritten und vierten Zeile:
    \[
    \text{Neue dritte Zeile: } Z_3 - 2Z_1 = \begin{pmatrix} 0 & -8 & 3 & 5 \end{pmatrix}.
    \]
    \[
    \text{Neue vierte Zeile: } Z_4 - Z_1 = \begin{pmatrix} 0 & -2 & 1 & 1 \end{pmatrix}.
    \]
    Die neue Matrix lautet:
    \[
    A_1 = \begin{pmatrix}
    1 & 2 & 0 & -3 \\
    0 & -1 & 1 & 0 \\
    0 & -8 & 3 & 5 \\
    0 & -2 & 1 & 1
    \end{pmatrix}.
    \]

    \item Nun wählen wir die zweite Zeile als Pivotzeile und eliminieren die Einträge in der zweiten Spalte der dritten und vierten Zeile:
    \[
    \text{Neue dritte Zeile: } Z_3 - 8Z_2 = \begin{pmatrix} 0 & 0 & -5 & 5 \end{pmatrix}.
    \]
    \[
    \text{Neue vierte Zeile: } Z_4 - 2Z_2 = \begin{pmatrix} 0 & 0 & -1 & 1 \end{pmatrix}.
    \]
    Die neue Matrix lautet:
    \[
    A_2 = \begin{pmatrix}
    1 & 2 & 0 & -3 \\
    0 & -1 & 1 & 0 \\
    0 & 0 & -5 & 5 \\
    0 & 0 & -1 & 1
    \end{pmatrix}.
    \]

    \item Schließlich verwenden wir die dritte Zeile als Pivotzeile und eliminieren den Eintrag in der dritten Spalte der vierten Zeile:
    \[
    \text{Neue vierte Zeile: } Z_4 - \frac{1}{5} Z_3 = \begin{pmatrix} 0 & 0 & 0 & 0 \end{pmatrix}.
    \]
    Die resultierende Matrix in Zeilenstufenform lautet:
    \[
    A' = \begin{pmatrix}
    1 & 2 & 0 & -3 \\
    0 & -1 & 1 & 0 \\
    0 & 0 & -5 & 5 \\
    0 & 0 & 0 & 0
    \end{pmatrix}.
    \]
\end{enumerate}

\textbf{Schritt 2: Interpretation der Zeilenstufenform.}

Die Zeilen 1, 2 und 3 sind linear unabhängig, da keine Zeile als Linearkombination der anderen dargestellt werden kann. Daher bilden sie eine Basis des Unterraums \( U \).

\textbf{Ergebnis:} Eine Basis von \( U \) ist gegeben durch:
\[
\mathcal{B}_U = \left\{
\begin{pmatrix} 1 \\ 2 \\ 0 \\ -3 \end{pmatrix}, \,
\begin{pmatrix} 0 \\ -1 \\ 1 \\ 0 \end{pmatrix}, \,
\begin{pmatrix} 0 \\ 0 \\ -5 \\ 5 \end{pmatrix}
\right\}.
\]

Der Unterraum \( U \) hat die Dimension \( 3 \).

\textbf{Teil (b):} \\
Wir bestimmen die Basen von \( W \cap W' \) und \( W + W' \).

\subsection*{Schritt 1: Basis von \( W \cap W' \)}

Der Schnitt \( W \cap W' \) besteht aus allen Vektoren, die sowohl in \( W \) als auch in \( W' \) liegen. Diese Vektoren sind Linearkombinationen der Basisvektoren von \( W \) und gleichzeitig Linearkombinationen der Basisvektoren von \( W' \). Wir formulieren dies als Gleichungssystem:

\[
x_1 \begin{pmatrix} 2 \\ 3 \\ -1 \\ 1 \end{pmatrix} + x_2 \begin{pmatrix} 0 \\ 1 \\ 0 \\ 0 \end{pmatrix} + x_3 \begin{pmatrix} 3 \\ 6 \\ -2 \\ 2 \end{pmatrix}
= y_1 \begin{pmatrix} 1 \\ 0 \\ 0 \\ 4 \end{pmatrix} + y_2 \begin{pmatrix} 4 \\ 0 \\ 0 \\ 1 \end{pmatrix}.
\]

Dies ergibt ein Gleichungssystem für die Einträge der beiden Seiten. Schreiben wir die Gleichung komponentenweise auf:

\[
\begin{pmatrix} 2x_1 + 0x_2 + 3x_3 \\ 3x_1 + x_2 + 6x_3 \\ -x_1 + 0x_2 - 2x_3 \\ x_1 + 0x_2 + 2x_3 \end{pmatrix}
=
\begin{pmatrix} y_1 + 4y_2 \\ 0y_1 + 0y_2 \\ 0y_1 + 0y_2 \\ 4y_1 + y_2 \end{pmatrix}.
\]

Das System ist:

1. \( 2x_1 + 3x_3 = y_1 + 4y_2 \),
2. \( 3x_1 + x_2 + 6x_3 = 0 \),
3. \( -x_1 - 2x_3 = 0 \),
4. \( x_1 + 2x_3 = 4y_1 + y_2 \).

\textbf{Lösung durch Gauß-Eliminationsverfahren:}

1. Aus Gleichung (3) folgt \( x_1 = -2x_3 \).
2. Setzen wir \( x_1 = -2x_3 \) in Gleichung (1) ein:
   \[
   2(-2x_3) + 3x_3 = y_1 + 4y_2 \quad \Rightarrow \quad -4x_3 + 3x_3 = y_1 + 4y_2 \quad \Rightarrow \quad -x_3 = y_1 + 4y_2.
   \]
3. Aus Gleichung (4) folgt, nach Einsetzen von \( x_1 = -2x_3 \):
   \[
   -2x_3 + 2x_3 = 4y_1 + y_2 \quad \Rightarrow \quad 0 = 4y_1 + y_2.
   \]
   Daraus folgt \( y_2 = -4y_1 \).

Zusammen mit \( x_1 = -2x_3 \) und \( x_3 = -y_1 - 4y_2 \) (aus Gleichung 1) lässt sich zeigen, dass alle Lösungen eine eindeutige Linearkombination ergeben. Der einzige Schnittpunkt ist:

\[
W \cap W' = \text{Spann} \left\{ \begin{pmatrix} 1 \\ 0 \\ 0 \\ 4 \end{pmatrix} \right\}.
\]

\textbf{Basis von \( W \cap W' \):}
\[
\mathcal{B}_{W \cap W'} = \left\{ \begin{pmatrix} 1 \\ 0 \\ 0 \\ 4 \end{pmatrix} \right\}.
\]

\subsection*{Schritt 2: Basis von \( W + W' \)}

Der Unterraum \( W + W' \) besteht aus allen Linearkombinationen der Vektoren in \( W \) und \( W' \). Die Basen von \( W \) und \( W' \) werden vereinigt, und wir überprüfen die lineare Unabhängigkeit der resultierenden Menge. Die Vektoren sind:

\[
\begin{pmatrix} 2 \\ 3 \\ -1 \\ 1 \end{pmatrix}, \,
\begin{pmatrix} 0 \\ 1 \\ 0 \\ 0 \end{pmatrix}, \,
\begin{pmatrix} 3 \\ 6 \\ -2 \\ 2 \end{pmatrix}, \,
\begin{pmatrix} 1 \\ 0 \\ 0 \\ 4 \end{pmatrix}, \,
\begin{pmatrix} 4 \\ 0 \\ 0 \\ 1 \end{pmatrix}.
\]

Wir bilden eine Matrix mit diesen Vektoren als Zeilen und bringen sie in Zeilenstufenform:

\[
A = \begin{pmatrix}
2 & 3 & -1 & 1 \\
0 & 1 & 0 & 0 \\
3 & 6 & -2 & 2 \\
1 & 0 & 0 & 4 \\
4 & 0 & 0 & 1
\end{pmatrix}.
\]

Durch Gauß-Elimination erhalten wir:

\[
A' = \begin{pmatrix}
1 & 0 & 0 & 0 \\
0 & 1 & 0 & 0 \\
0 & 0 & 1 & 0 \\
0 & 0 & 0 & 1 \\
0 & 0 & 0 & 0
\end{pmatrix}.
\]

Die ersten vier Zeilen sind linear unabhängig. Daher ist eine Basis von \( W + W' \):

\[
\mathcal{B}_{W + W'} = \left\{
\begin{pmatrix} 2 \\ 3 \\ -1 \\ 1 \end{pmatrix}, \,
\begin{pmatrix} 0 \\ 1 \\ 0 \\ 0 \end{pmatrix}, \,
\begin{pmatrix} 1 \\ 0 \\ 0 \\ 4 \end{pmatrix}, \,
\begin{pmatrix} 4 \\ 0 \\ 0 \\ 1 \end{pmatrix}
\right\}.
\]

\textbf{Ergebnis:}
\begin{enumerate}
    \item Eine Basis von \( W \cap W' \) ist:
    \[
    \mathcal{B}_{W \cap W'} = \left\{ \begin{pmatrix} 1 \\ 0 \\ 0 \\ 4 \end{pmatrix} \right\}.
    \]
    \item Eine Basis von \( W + W' \) ist:
    \[
    \mathcal{B}_{W + W'} = \left\{
    \begin{pmatrix} 2 \\ 3 \\ -1 \\ 1 \end{pmatrix}, \,
    \begin{pmatrix} 0 \\ 1 \\ 0 \\ 0 \end{pmatrix}, \,
    \begin{pmatrix} 1 \\ 0 \\ 0 \\ 4 \end{pmatrix}, \,
    \begin{pmatrix} 4 \\ 0 \\ 0 \\ 1 \end{pmatrix}
    \right\}.
    \]
\end{enumerate}


\section*{Aufgabe 4}

Seien \( f: U \to V \) und \( g: V \to W \) lineare Abbildungen zwischen \( K \)-Vektorräumen. Zeigen Sie: Ist \( g \circ f \) ein Isomorphismus, so sind \( \ker(g) \) und \( \operatorname{im}(f) \) Komplementärräume in \( V \).

\subsection*{Lösung}

\textbf{Beweis:} Wir zeigen, dass \( \ker(g) \oplus \operatorname{im}(f) = V \), wenn \( g \circ f \) ein Isomorphismus ist.

\textbf{1. Eigenschaften eines Isomorphismus:} \\
Da \( g \circ f \) ein Isomorphismus ist, gilt:
\begin{enumerate}
    \item \( g \circ f \) ist bijektiv, also sowohl injektiv als auch surjektiv.
    \item Für \( f \) gilt: \( \ker(f) = \{0\} \), da \( f \) injektiv sein muss, damit \( g \circ f \) injektiv ist.
    \item Für \( g \) gilt: \( \operatorname{im}(g) = W \), da \( g \) surjektiv sein muss, damit \( g \circ f \) surjektiv ist.
\end{enumerate}

\textbf{2. Zerlegung von \( V \):} \\
Sei \( v \in V \). Wir wollen \( v \) als Summe \( v = v_1 + v_2 \) mit \( v_1 \in \ker(g) \) und \( v_2 \in \operatorname{im}(f) \) schreiben. Dafür nutzen wir die Eigenschaften von \( f \) und \( g \).

\begin{enumerate}
    \item Sei \( v_2 = f(u) \) für ein \( u \in U \). Da \( g \circ f \) bijektiv ist, existiert zu jedem \( w \in W \) ein eindeutiges \( u \in U \), sodass \( g(f(u)) = w \). Insbesondere ist \( \operatorname{im}(f) \subseteq V \).
    \item Sei \( v_1 \in \ker(g) \). Per Definition von \( \ker(g) \) gilt \( g(v_1) = 0 \). Da \( \operatorname{im}(f) \) surjektiv auf \( W \) wirkt, ist jeder \( v \in V \) eindeutig als Summe \( v = v_1 + v_2 \) darstellbar mit \( v_1 \in \ker(g) \) und \( v_2 \in \operatorname{im}(f) \).
\end{enumerate}

\textbf{3. Komplementarität:} \\
Um zu zeigen, dass \( \ker(g) \) und \( \operatorname{im}(f) \) Komplementärräume sind, müssen zwei Eigenschaften erfüllt sein:
\begin{enumerate}
    \item \( \ker(g) \cap \operatorname{im}(f) = \{0\} \): \\
    Sei \( v \in \ker(g) \cap \operatorname{im}(f) \). Dann gilt \( v \in \ker(g) \), also \( g(v) = 0 \), und gleichzeitig \( v = f(u) \) für ein \( u \in U \). Da \( g \circ f \) injektiv ist, folgt \( u = 0 \), also \( v = 0 \).
    \item \( \ker(g) + \operatorname{im}(f) = V \): \\
    Sei \( v \in V \). Da \( g \circ f \) surjektiv ist, existiert ein \( u \in U \) mit \( f(u) \in \operatorname{im}(f) \). Somit kann jedes \( v \in V \) als Summe eines Elements aus \( \ker(g) \) und \( \operatorname{im}(f) \) geschrieben werden.
\end{enumerate}

\textbf{Schlussfolgerung:} \\
Da \( \ker(g) \cap \operatorname{im}(f) = \{0\} \) und \( \ker(g) + \operatorname{im}(f) = V \), folgt:
\[
\ker(g) \oplus \operatorname{im}(f) = V.
\]

\textbf{Ergebnis:} Sind \( f: U \to V \) und \( g: V \to W \) linear und ist \( g \circ f \) ein Isomorphismus, so sind \( \ker(g) \) und \( \operatorname{im}(f) \) Komplementärräume in \( V \).


\bibliography{main}
\bibliographystyle{plain}

\end{document}
