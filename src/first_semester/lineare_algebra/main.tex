%! Author = felix
%! Date = 11.10.2024

% Preamble
\documentclass[11pt]{article}

% Packages
\usepackage{amsmath}
\usepackage{amsfonts}

% Document
\begin{document}

    \section{Übugnen zur Vorlesung Lineare Algebra - Übungsblatt 1}

    \subsection{Aufgabe 1.1}
    Aufgabe: Bestimmmen Sie ob die folgenden Teilmengen von (R^2) Loesungsmengen linearer Gleichungssysteme sind.
    Lösung:
        1) Da lineare Gleichungssysteme wie der Name schon sagt, linear sind, muss der zusammenhang zwischen den Variablen der Lösungsmenge auch linear sein.

        2) Lineare Gleichungssysteme können grundsätzlich drei verschiedene Lösungsmengen haben (Hier visualisiert im R^2 Raum):
            1. Keine Lösung ( Geraden im R^2 Raum verlaufen parallel zueinander, sind aber nicht dieselbe Gerade=
            2. Eine Lösung ( Geraden schneiden sich in einem Punkt. )
            3. Unendlich viele Lösungen (Geraden sind identisch.)

    a) $\left\{ \begin{pmatrix} t \\ 2t+1 \end{pmatrix} \in \mathbb{R}^2 : t \in \mathbb{R} \right\} \subset \mathbb{R}^2$







    \bibliography{main}
    \bibliographystyle{plain}

\end{document}
